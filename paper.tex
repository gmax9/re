\PassOptionsToPackage{table}{xcolor}
\documentclass[manuscript,nonacm]{acmart}
\AtBeginDocument{%
  \providecommand\BibTeX{{%
    Bib\TeX}}}

\usepackage{amsmath}
\let\Bbbk\relax
\usepackage{amssymb}
\let\Bbbk\relax
\usepackage{pifont}
\usepackage{wasysym}
\usepackage{marvosym}
\usepackage{threeparttable}
\usepackage{textgreek}
\usepackage{paralist}
\usepackage{filecontents}
\usepackage{booktabs}
\usepackage{multirow}
\usepackage{tikz,siunitx}
\usetikzlibrary{shapes.geometric,shapes.symbols}
\usepackage{threeparttable}
\usepackage{enumerate}
\usepackage{comment}

% Commands for Surveyed Works Table
\newcommand{\cmark}{\ding{51}}%
\newcommand{\xmark}{\ding{55}}%
\newcommand{\markA}{\ding{66}}%
\newcommand{\markB}{\ding{71}}%
\newcommand{\markC}{\ding{75}}%
\newcommand{\markD}{\ding{168}}%
\newcommand{\markE}{\ding{169}}%
\newcommand{\markF}{\ding{170}}%
\newcommand{\markG}{\ding{171}}%
\newcommand{\markH}{\ding{92}}%
\newcommand{\markI}{\ding{214}}%
\newcommand{\markJ}{\ding{166}}%
\newcommand{\markX}{\Sagittarius} % heh
\newcommand{\markY}{\Virgo}
\newcommand{\markZ}{\Moon}
\newcommand{\markEtc}{\textbf{?}}
\def\rot{\rotatebox}
\newcommand*\circled[1]{\tikz[baseline=-3pt]{
            \node (X) [shape=circle,scale=0.5,fill=black,text=white,font=\bfseries, text centered, draw,inner sep=0pt] {\strut #1};}}

\newcommand{\wc}[1]{\textit{\textcolor{magenta}{#1}}} % Word-Choice macro

\begin{document}

\title{PLACEHOLDER}
\author{Max Gao}
% \authornote{Both authors contributed equally to this research.}
\affiliation{%
  \institution{CAIDA/UC San Diego}
  \city{San Diego}
  \state{CA}
  \country{USA}
}
\email{magao@ucsd.edu}

\begin{abstract}
Analysis of network telescope, or darknet, traffic has provided substantial visibility into the security and availability of the Internet for decades.
To operationalize darknet monitoring capabilities, researchers have proposed numerous frameworks that automate the detection of Internet-wide events.
Despite an abundance of available frameworks, less attention has been directed towards systematic comparison of their capabilities. 
As a result, \dots
In this report, we survey existing work that proposes and evaluates such frameworks on real-world darknet traffic.
We review tasks that motivate the techniques each framework employs as well as experimental details of their evaluation; and finally, discuss future \dots
% To more effectively operationalize network telescopes for monitoring the security and availability of the Internet, researchers have proposed a diverse body of frameworks which codify empirical post-hoc analysis techniques.
% Yet, \dots
\end{abstract}

\maketitle

\section{Introduction}

\begin{enumerate}
  \item Overview of topic - 2 sentences.
  \item 
  \item Summary of contributions - 3 sentences.
\end{enumerate}

% Network telescopes, or darknets, have been instrumental to both researchers and practitioners due to their capacity to observe Internet-wide phenomena in the unsolicited traffic they receive.
% Over the past two decades, research efforts have shifted from characterizing such phenomena and their observable signals towards translating these empirical insights into operationalizable frameworks for monitoring the Internet's security and availability in near real-time.

Network telescopes, or darknets, have been instrumental to both researchers and practitioners due to their capacity to observe Internet-wide phenomena in the unsolicited traffic they receive.
Over the past two decades, research efforts have shifted from characterizing such phenomena and their observable signals towards translating empirical insights into operationalizable frameworks for monitoring the Internet's security and availability in near real-time.
As a result, a large variety of frameworks have been proposed. 
Each share the same general goal of automating event detection.
They vary widely in their choice of techniques and selection of darknet traffic features as they are designed with different event definitions in mind.
\textit{discuss why it's difficult to speculate on their performance without further empirical evaluation.}


% Despite the variety of available frameworks for darknet traffic analysis, selecting the most suitable framework off-the-shelf for a given darknet is challenging.
% While the variety of frameworks is frank

% Traffic Growth
% Complexity Growth - new 
% 

We begin this report with an overview of network telescopes, the traffic they collect, and the types of Internet-wide events inferrable through analysis of such traffic.
We then survey various frameworks that have been proposed to automate event detection and construct a taxonomy that compares their 1) analysis goals and designs; and 2) performance as informed by empirical evaluations on real-world darknet traffic.
Using this taxonomy, we identify challenges and opportunities for future research where efforts may yield improvements over the current state-of-art.

\section{Background}

\subsection{Network Telescopes}

% Brief overview of darknets and their traffic (IBR)
Network telescopes, or darknets, consist of routable portions of IPv4 address space that receive but do not respond to unsolicited Internet traffic. 
Passive collection of this unidirectional traffic has enabled extensive research into various Internet-wide activity ranging from 
malicious scanning~\cite{@@}, residual backscatter from DoS attacks~\cite{@@}, widespread outages~\cite{@@}, and non-trivial network misconfigurations~\cite{@@}.

% Discuss IBR, how it's been analyzed, practical use-cases for security / outage detection



\section{Darknet Event Detection Frameworks}

In this section, we provide a broad survey of existing darknet event detection frameworks that exemplify a wide spectrum of 
different analytical tasks and techniques; in addition, we review the empirical evaluations of these frameworks to consider 
their efficacy in practice.
We structure our survey by categorizing proposed frameworks primarily based on the \textit{type of event} (\textit{e.g.}, \textit{scanning}, 
\textit{backscatter}, and \textit{outages}) they are designed to detect.
Within these categories, we review and emphasize two important aspects of each individual framework:
1) the \textit{classes of general techniques} that encompass their employed algorithms; and
2) the \textit{event schema} or unit of detection.
We define 7 classes of techniques that differentiate the functions of algorithms within the scope of each framework: 
\textit{dimensionality reduction}, \textit{clustering}, \textit{forecasting}, \textit{thresholding}, \textit{representation learning},
\textit{frequent pattern mining}, and \textit{fingerprinting}.
Some frameworks by design are capable of detecting multiple types of events using a combination of different techniques.

% We structure our survey of proposed frameworks by grouping 
% by categorizing frameworks into groups based primarily 
% 1) the primary type of darknet event they were designed to detect; and 
% 2) their definition of 

\subsection{Scanning}

Frameworks designed to detect scanning events in darknet traffic typically base their unit of detection on either source or destination attributes.
Source-based event schemas enable discernment of different types of sender behavior within incoming traffic to a darknet whereas in contrast, 
destination-based event schemas identify specific targets that incoming traffic is directed towards.

% Enummerate works here
Sourced-based frameworks achieve detection at a variety of resolutions, ranging from individual and subnetted addresses to the level of autonomous-systems (ASes).
Several of these leverage representation learning.
Gioacchini et al.~\cite{2023gioacchini,2021gioacchini} propose a domain-specific adaptation of Word2Vec~\cite{2013mikolov} that detect anomalous clusters of senders from packet arrival sequences.
Kallitsis et al.~\cite{2022kallitsis} trained autoencoders on a more comprehensive set of features and clustered senders using their lower-dimensional embeddings.


\subsection{Backscatter}

\subsection{Outages}


% We review individual components of each framework, considering their impact on \textit{the high-level }
% In addition, we include 
% In this section, we provide a broad survey of existing darknet event detection frameworks that we consider in this report.
% In addition to frameworks themselves, we consider characteristics of their empirical evaluation.

% Table~\ref{tab:frameworks} lists publications we collected, grouped by their authors; groups that consist of multiple works 
% indicate works by similar authors that share significant overlap in their framework design, e.g., proof-of-concepts paired with their mature versions.
% \dots


% \noindent{\textbf{Activity Type(s)}} refer to one or multiple types of events, (\textit{e.g.}, \textit{scanning}, \textit{backscatter}, \textit{outages}, or \textit{misconfigurations}) that a framework is capable of capturing.
% While in practice these are dependent on traffic inputs, we determine detectable activities for each framework based on explicitly stated design goals and evaluation results.

% \noindent{\textbf{Technique(s)}} encompass the specific algorithms employed by each framework and differentiate general functions frameworks leverage, often in combination, to
% achieve their detection objectives. We define 7 classes of techniques which include \textit{dimensionality reduction}, \textit{clustering}, \textit{forecasting}, \textit{thresholding}, \textit{representation learning},
% \textit{frequent pattern mining}, and \textit{fingerprinting}.

% \textbf{Algorithm(s)} 

\begin{center}
    \rowcolors{2}{gray!50}{white}
    \begin{table}[]
        \small
        % \scriptsize
        \caption{Characteristics of event detection frameworks proposed in our surveyed literature.}
        \label{tab:frameworks}
        \begin{tabular}{lllclc}
            \toprule
            \multicolumn{1}{c}{Work} & \multicolumn{5}{c}{Framework} \\
            \midrule
            & \rot{45}{i. Activity} & \rot{45}{ii. Technique(s)} & \rot{45}{iii. Algorithm(s)} & \rot{45}{\begin{tabular}[c]{@{}l@{}}iv. Traffic\\ Rep.\end{tabular}} & \rot{45}{\begin{tabular}[c]{@{}l@{}}v. Output\\ Attributes\end{tabular}} \\
            \midrule
            Evrard et al.~\cite{2019evrard}                        & \markX                    & \markB        & Dijkstra's Algorithm~\cite{1959dijkstra}, K-Nearest Neighbors~\cite{1967cover,1989fix} & \markD & Dst. Port \\
            Lagraa et al.~\cite{2017lagraa,2019lagraa}             & \markX                    & \markB        & Louvain Algorithm~\cite{2006newman,2008blondel} & \markD & Dst. Port \\
            Kallitsis et al.~\cite{2022kallitsis}                  & \markX\markY              & \markI\markA\markB  & Autoencoder Dimensionality Reduction~\cite{2006hinton}, K-Means~\cite{1967macqueen} & \markE & Src. IP   \\
            Iglesias et al.~\cite{2019iglesias}                    & \markX\markY              & \markB\markH  & K-Medoids~\cite{2009park}, Fuzzy-Gustafson~\cite{1999krishnapuram}, MAD-Thresholding~\cite{2004liu} & \markE & Src. IP   \\
            Nishikaze et al.~\cite{2015nishikaze}                  & \markX                    & \markB        & Hierarchical Clustering           & \markE & Src. /16  \\
            Soro et al.~\cite{2020soro}                            & \markX                    & \markB        & Louvain Algorithm~\cite{2006newman,2008blondel} & \markF & Src. AS   \\
            % Cohen et al.~\cite{2020cohen}                          & \markX          & \markA\markB  & Word2vec, DBSCAN & \markF & Dst. Port \\
            Gioacchini et al.~\cite{2021gioacchini,2023gioacchini} & \markX                    & \markA\markB  & Word2Vec~\cite{2013mikolov}, K-Means~\cite{1967macqueen}, K-Nearest Neighbors~\cite{1967cover,1989fix}, Louvain Algorithm~\cite{2006newman,2008blondel} & \markF & Src. IP   \\
            Han et al.~\cite{2021han,2022han}                      & \markX                    & \markA        & NMF~\cite{2000lee}    & \markG & Src. /16, Dst. Port \\
            Han et al.~\cite{2020han,2022han}                      & \markX\markEtc            & \markA        & GLASSO~\cite{2008friedman} & \markG & Src. /16  \\
            Kanehara et al.~\cite{2019kanehara,2022han}            & \markX                    & \markA\markH  & LRA-NTD~\cite{2015zhou}, FTSD~\cite{2010caiafa}, Otsu-Thresholding~\cite{1979otsu}          & \markG & Src. /16, Dst. Port \\
            Kartsioukas et al.~\cite{2023kartsioukas}              & \markX                    & \markA\markH  & Incremental PCA~\cite{2012arora} & \markG & \textit{TODO} \\
            Ban et al.~\cite{2016ban}                              & \markX                    & \markA        & Frequent Pattern Mining~\cite{2000han,2007han}, Hierarchical Clustering~\cite{2012murtagh}  & \markF\markG & Dst. Port \\ 
            Torabi et al.~\cite{2020torabi,2018torabi}             & \markX                    & \markJ        & Association Rule Mining~\cite{1993agrawal}, DBSCAN~\cite{1996ester} & \markD,\markE & \textit{TODO} \\
            Tanaka et al.~\cite{2023tanaka,2021tanaka}             & \markX                    & \textit{TODO} & \textit{TODO} & Source \\
            Niranjana et al.~\cite{2019niranjana}                  & \markX                    & \markA\markB  & PCA~\cite{1901pearson,1993hotelling} &  Source \\
            Cabana et al.~\cite{2019cabana}                        & \markX                    & \markB\markH  & Conduction Detection Algorithm~\cite{2015lu}, Fastcluster~\cite{2013mullner} & \markD\markE & \textit{TODO} \\
            Shaikh et al.~\cite{2018shaikh}                        & \markX\markY\markZ\markEtc & \textit{TODO} & AdaBoost~\cite{@@}, Gradient Boosting~\cite{@@}, Random Forest~\cite{@@}     & \markE       & \textit{TODO} \\  
            Collins et al.~\cite{2007collins}                      & \markX                    & \markH        & Union-Find Algorithm~\cite{1991galil} & \markD & \textit{TODO} \\ 
            % Ban et al.~\cite{2017ban}                              & \markX          & \markH        & CUMSUM & \markG & \\
            % Bou-Harb et al.~\cite{bouharb2013dfa,2014bouharb}      &  & & & & \\
            % & & & & 
            %  & Src.\& Dst. IP, Src.\&Dest Port,\\
            % \rowcolor{gray!50}
            %  \multirow{-2}{*}{Bou-Harb et al.~\cite{2019bouharb,2015bouharb}}    & \multirow{-2}{*}{\markX\markEtc}  & \multirow{-2}{*}{\markC\markB}  & &  \multirow{-2}{*}{\markE}& Protocol, TTL, Flags\\
            %  \rowcolor{white}
            Zakroum et al.~\cite{2022zakroum,2018zakroum}          & \markX          & \markC\markB  & Spectral Clustering~\cite{2001ng}, LSTM~\cite{1997hochreiter}       & \markG        & Dst. Port \\
            \bottomrule
            \multicolumn{6}{l}{Activities: \markX-Scanning, \markY-DDoS, \markZ-Outage, \markEtc-Misconfiguration} \\
            \rowcolor{white}
            \multicolumn{6}{l}{Techniques:\markA-Dimensionality Reduction, \markB-Clustering, \markC-Forecasting, \markH-Thresholding, \markI-Representation Learning, \markJ-Frequent Pattern Mining, \markH-Fingerprinting} \\
            \multicolumn{6}{l}{Traffic representation: \markD-Graph, \markE-Feature Vector, \markF-Sequences, \markG-Time Series}
        \end{tabular}
    \end{table}
\end{center}


\section{Section}


\section{Acknowledgments}
asadfadsf

\section{Appendices}

asdfasdf

\begin{acks}
To Robert, for the bagels and explaining CMYK and color spaces.
\end{acks}

\bibliographystyle{plain}
\bibliography{bib/refs.bib}

\appendix
\section{Research Methods}

\subsection{Part One}

Lorem ipsum dolor sit amet, consectetur adipiscing elit. Morbi
malesuada, quam in pulvinar varius, metus nunc fermentum urna, id
sollicitudin purus odio sit amet enim. Aliquam ullamcorper eu ipsum
vel mollis. Curabitur quis dictum nisl. Phasellus vel semper risus, et
lacinia dolor. Integer ultricies commodo sem nec semper.

\end{document}
\endinput