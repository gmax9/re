\documentclass[12pt]{article}

% ----- Packages -----
\usepackage{amsmath} % For advanced math typesetting
\usepackage{amssymb} % For additional math symbols
\usepackage{amsthm} % For theorem-like environments
\usepackage{graphicx} % For including figures
\usepackage[margin=1in]{geometry} % For setting margins
\usepackage[utf8]{inputenc} % For handling special characters
\usepackage{url} % For typesetting URLs correctly
\usepackage{hyperref} % For creating hyperlinks in the document
\usepackage{natbib} % For handling citations. You may need to change this depending on the required citation style.
\usepackage{caption} % For better control over figure captions
\usepackage{enumitem} % For more control over lists

% ----- Custom commands and settings -----
\newcommand{\myname}{Your Name}
\newcommand{\myadvisor}{Your Advisor's Name}
\newcommand{\examdate}{September 12, 2025}
\newcommand{\examtitle}{Title of Your Research Exam}

% ----- Document body -----
\begin{document}

% Title Page
\begin{titlepage}
    \centering
    \vspace*{\stretch{1}}
    {\Large\bfseries \examtitle\par}
    \vspace{1.5cm}
    {\large A Research Exam Submitted to the\par}
    {\large Department of Computer Science and Engineering\par}
    {\large University of California, San Diego\par}
    \vspace{2cm}
    {\large by\par}
    \vspace{0.5cm}
    {\Large\bfseries \myname\par}
    \vspace{2cm}
    {\large Exam Committee:\par}
    {\large\myadvisor, Chair\par}
    {\large [Committee Member 1]\par}
    {\large [Committee Member 2]\par}
    {\large [Committee Member 3]\par}
    \vfill
    {\large \examdate\par}
\end{titlepage}

% Table of Contents
\tableofcontents
\newpage

% Main content
\section{Introduction}
\label{sec:introduction}
This document outlines the research presented for the UCSD Computer Science and Engineering research exam. It covers a comprehensive review of the relevant literature and proposes new research directions.

\section{Background and Related Work}
\label{sec:related-work}
This section provides a summary of the foundational concepts and a critical review of existing literature relevant to the research topic. Proper citation is crucial here. For example, \cite{citation1, citation2}.

\subsection{Literature Review}
The literature review should be a detailed analysis of key papers in your field.

\section{Proposed Research}
\label{sec:proposed-research}
This section details the proposed research plan, including methodologies and expected outcomes.

\section{Discussion and Future Work}
\label{sec:discussion}
Discuss the implications of your proposed work and potential avenues for future research.

% Bibliography
\bibliographystyle{plainnat} % Or another style, e.g., abbrvnat, unsrtnat
\bibliography{bibs/refs.bib} % Use 'myreferences.bib' for your bibliography file

\end{document}