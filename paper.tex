\PassOptionsToPackage{table}{xcolor}
\documentclass[manuscript,nonacm]{acmart}
% \documentclass[sigconf,anonymous]{acmart}
\AtBeginDocument{%
  \providecommand\BibTeX{{%
    Bib\TeX}}}

\usepackage{amsmath}
\let\Bbbk\relax
\usepackage{amssymb}
\let\Bbbk\relax
\usepackage{pifont}
\usepackage{wasysym}
\usepackage{marvosym}
\usepackage{threeparttable}
\usepackage{textgreek}
\usepackage{paralist}
\usepackage{filecontents}
\usepackage{booktabs}
\usepackage{multirow}
\usepackage{tikz,siunitx}
\usetikzlibrary{shapes.geometric,shapes.symbols}
\usepackage{threeparttable}
\usepackage{enumerate}
\usepackage{comment}
\usepackage{arydshln}
\usepackage{enumitem}
\usepackage{makecell}

\newcommand{\mybox}[4]{
    \begin{figure}[h]
        \centering
    \begin{tikzpicture}
        \node[anchor=text,text width=\columnwidth-1.5em, draw, rounded corners, line width=1pt, fill=#3, inner sep=1em, inner ysep=0.5em] (big) {\\#4};
        % \node[draw, rounded corners, line width=.5pt, fill=#2, anchor=west, xshift=5mm] (small) at (big.north west) {#1};
    \end{tikzpicture}
    \end{figure}
}

% Commands for Surveyed Works Table
\newcommand{\cmark}{\ding{51}}%
\newcommand{\xmark}{\ding{55}}%
\newcommand{\markA}{\ding{66}}%
\newcommand{\markB}{\ding{71}}%
\newcommand{\markC}{\ding{75}}%
\newcommand{\markD}{\ding{168}}%
\newcommand{\markE}{\ding{169}}%
\newcommand{\markF}{\ding{170}}%
\newcommand{\markG}{\ding{171}}%
\newcommand{\markH}{\ding{92}}%
\newcommand{\markI}{\ding{214}}%
\newcommand{\markJ}{\ding{166}}%
\newcommand{\markX}{\Sagittarius} % heh
\newcommand{\markY}{\Virgo}
\newcommand{\markZ}{\Moon}
\newcommand{\markEtc}{\textbf{?}}
\def\rot{\rotatebox}
\newcommand*\circled[1]{\tikz[baseline=-3pt]{
            \node (X) [shape=circle,scale=0.5,fill=black,text=white,font=\bfseries, text centered, draw,inner sep=0pt] {\strut #1};}}
\newcommand{\notcheckmark}{{$\surd$}\textsuperscript{\textcolor{black}{\kern-0.35em{\bf--}}}}
            
\newcommand{\revise}[1]{\textit{\textcolor{red}{#1}}}
\newcommand{\wc}[1]{\textit{\textcolor{magenta}{#1}}} % Word-Choice macro
\newcommand{\maxnote}[1]{\textit{\textcolor{violet}{#1 --Max}}}
\newcolumntype{P}[1]{>{\centering\arraybackslash}p{#1}}

\begin{document}

\title{A Survey of Darknet Detection Methodologies: Design, Implementation, and Assessment}
\author{Max Gao}
% \authornote{Both authors contributed equally to this research.}
\affiliation{%
  \institution{CAIDA/UC San Diego}
  \city{San Diego}
  \state{CA}
  \country{USA}
}
\email{magao@ucsd.edu}

\begin{abstract}
Network telescope (darknet) traffic has been instrumental in providing visibility into Internet-wide phenomena related to security and availability such as malicious scanning, denial-of-service (DoS) backscatter, and outages. 
To better operationalize darknets for their monitoring capabilities, researchers have proposed a number of automated detection methodologies which span a diverse set of analytical techniques applied to different features of darknet traffic. 
Yet despite the abundance of methods, a comprehensive view of their comparative capabilities remains unclear due to gaps across their implementations and inconsistencies across their empirical assessments. 
In this report, we review the evolution of darknet event detection methodologies over the past decade and examine in-depth the challenges that face a comprehensive, systematic assessment. We conclude with a discussion of future directions to address these challenges.
% Yet despite the abundance of methods, their comparative capabilities remain unclear due to limited labeled datasets, difficultings in replicating their implementations, and the lack of a unified validation approach.
% In this report, we review the evolution of darknet detection methodologies over the past decade and examine in-depth the challenges that face a comprehensive, systematic assessment.
% We conclude with a discussion of future directions for improving assessment rigor and method comparability.
    
% Analysis of network telescope, or darknet, traffic has provided substantial visibility into the security and availability of the Internet for decades.
% To operationalize darknet monitoring capabilities, researchers have proposed numerous frameworks that automate the detection of Internet-wide events.
% Despite an abundance of available frameworks, less attention has been directed towards systematic comparison of their capabilities. 
% As a result, \dots
% In this report, we survey existing work that proposes and evaluates such frameworks on real-world darknet traffic.
% We review tasks that motivate the techniques each framework employs as well as experimental details of their evaluation; and finally, discuss future \dots
% To more effectively operationalize network telescopes for monitoring the security and availability of the Internet, researchers have proposed a diverse body of frameworks which codify empirical post-hoc analysis techniques.
% Yet, \dots
\end{abstract}

\maketitle

\section{Introduction}

Network telescopes, or darknets, have been instrumental to both researchers and practitioners for their ability to observe Internet-wide phenomena in the unsolicited traffic they receive. 
Over the past two decades, research efforts have shifted from characterizing darknet phenomena and their observable signals towards translating such empirical insights into operationalizable methodologies for monitoring the Internet's security and availability in near real-time.
These efforts have culminated in a number of methodologies that, despite sharing a common goal of automatically detecting events from darknet traffic, 
differ widely in the technique(s) they employ (\textit{e.g.,} dimensionality reduction, forecasting, representation learning), features of the traffic they exploit (\textit{e.g.,} packet inter-arrival times, destination port sequences), and their operational definition of what constitutes an event (\textit{e.g.,} a shift in sender clusters, anomalous traffic volumes towards specific ports). 
As recent studies propose an increasing number of methods that leverage machine-learning techniques,
systematic assessment of their capabilities are needed to enable fair yet rigorous comparisons
to determine their effectiveness in-practice.
However as our report finds, many of these methods have been developed, implemented, and 
assessed independently by different research groups without necessary standard procedures to ensure
their comparability.
As a result, this lack of standardization discourages the development of new and existing methods
which we encountered in our prior work~\cite{2024gao}.
% However, many of these methodologies have been developed, implemented, and assessed independently by different research groups without a uniform standard for best-practices. 
% This fragmentation discourages the development of new or improvement to existing methodologies due to several challenges that we encountered in our prior work~\cite{2024gao} and later describe in this report.

In this report, we survey an extensive body of literature that proposes darknet event detection methodologies to investigate the extent that the challenges we encountered impact the development and assessments of darknet-event detection methodologies at large. 
Overall, we find a low degree of replicability among proposed methodologies and inconsistencies across datasets and validation approaches in their assessments.
Section 1 provides an overview of canonical literature that characterizes darknet traffic and its associated phenomena.
We discuss notable findings from past empirical studies and their role in monitoring darknet traffic. 
Section 2 introduces our taxonomy of detection methodologies with a breakdown of their detected events, employed techniques, and the features and representations of their traffic inputs alongside 
summarized intuitions guiding their design.
Section 3 reviews empirical assessments of prior work separate from methods themselves to highlight specific gaps that result from the lack of standard practices. 
Finally, we conclude our survey's findings with directions for future work intended 
to support more robust evidence-informed method selection for darknet event detection tasks 
by enabling systematic comparative assessments.



% As we experienced while designing and evaluating our prior work, DarkSim~\cite{2024gao}, this challenge takes form in limited replicability of implementations, inconsistent approaches to validation, and non-standardized metrics for measuring performance.

\begin{comment}
% Network telescopes, or darknets, have been instrumental to both researchers and practitioners for their ability to observe Internet-wide phenomena in the unsolicited traffic they receive. 
% Over the past two decades, research efforts have shifted from characterizing darknet phenomena and their observable signals towards translating such empirical insights into operationalizable methodologies for monitoring the Internet's security and availability in near real-time.
% This has resulted in a number of methodologies that each leverage different approaches to achieve a common goal of broadly automating darknet event detection. 
% \maxnote{see other papers for enumerating on specific details in intro} As security risks mount along growing volumes of darknet traffic, such methodologies represents a crucial and necessary transition from conventional post-hoc manual traffic analysis workflows. 
% However, the utility of these methodologies depend critically on their accuracy and validity which remain elusive to evaluate due to several challenges that we identify through our survey. 
% These efforts have resulted in a variety of methodologies that share the common goal of broadly automating event detection while differing widely in the algorithmic techniques they employ (e.g., statistical forecasting, matrix decompositions, and graph-based models), 
% the traffic features they exploit (e.g., packet counts, port sequences, temporal bursts), 
% and their operational definition of what constitutes an event (ranging from short-lived scans to sustained probing campaigns).
% As security risks mount along growing volumes of darknet traffic, such methodologies represent a crucial and necessary transition from post-hoc manual traffic analysis towards a more \maxnote{dots} means of detection.
% However, the utility of these frameworks depend critically on their accuracy and validity, which remain elusive to evaluate due challenges posed by \dots.

% However, as we identify in our suvey, methodologies differ widely in several ways: the algorithmic techniques they employ (e.g., statistical forecasting, matrix decompositions, and graph-based models), the traffic features they exploit (e.g., packet counts, port sequences, temporal bursts), 
% and their operational definition of what constitutes an event (ranging from short-lived scans to sustained probing campaigns).

% Network telescopes, or darknets, have been instrumental to both researchers and practitioners due to their capacity to observe Internet-wide phenomena in the unsolicited traffic they receive. 
% The visibility afforded by these observatories have enabled significant progress in the areas of cybersecurity and censorship, evident in a rich body of work that centers around the study of malicious scanning, denial-of-service attacks, and Internet outages. 
% The visibility afforded by these observatories have enabled significant progress in the areas of cybersecurity and censorship, evident in a rich body of work centered around 
% malicious scanning, denial-of-service events, and Internet outages. 

% Over the past two decades, research efforts have shifted from characterizing darknet phenomena and their observable signals towards translating such empirical insights into operationalizable frameworks for monitoring the Internet's security and availability in near real-time.  
% This has led to the development of a number of frameworks that each share the common goal of automating the detection of events in darknet traffic while taking notably different approaches to doing so\maxnote{feedback: elaborate on the specifics here.}.
% As security risks mount along with growing volumes of darknet traffic, such frameworks represent a crucial and necessary transition from conventional post-hoc manual traffic analysis workflows.
% However, the utility of these frameworks, particularly those more recent that leverage machine learning techniques, depend critically on their accuracy and validity which remain elusive to evaluate due to challenges posed by the diversity of framework detection objectives and approaches.

% As a result, a number of frameworks that each share a common goal of automating event detection within darknet traffic have been proposed.
% This has resulted in a number of proposed frameworks that share a common goal of automating the detection of events within darknet traffic.
% \begin{enumerate}
%   \item Overview of topic - 2 sentences.
%   \item 
%   \item Summary of contributions - 3 sentences.
% \end{enumerate}

% Network telescopes, or darknets, have been instrumental to both researchers and practitioners due to their capacity to observe Internet-wide phenomena in the unsolicited traffic they receive.
% Over the past two decades, research efforts have shifted from characterizing such phenomena and their observable signals towards translating these empirical insights into operationalizable frameworks for monitoring the Internet's security and availability in near real-time.

% Network telescopes, or darknets, have been instrumental to both researchers and practitioners due to their capacity to observe Internet-wide phenomena in the unsolicited traffic they receive.
% Over the past two decades, research efforts have shifted from characterizing such phenomena and their observable signals towards translating empirical insights into operationalizable frameworks for monitoring the Internet's security and availability in near real-time.
% As a result, a large variety of frameworks have been proposed. 
% Each share the same general goal of automating event detection.
% They vary widely in their choice of techniques and selection of darknet traffic features as they are designed with different event definitions in mind.
% \textit{discuss why it's difficult to speculate on their performance without further empirical evaluation.}

% Despite the variety of available frameworks for darknet traffic analysis, selecting the most suitable framework off-the-shelf for a given darknet is challenging.
% While the variety of frameworks is frank

% Traffic Growth
% Complexity Growth - new 
\end{comment}

% In this report, we survey an extensive body of literature that proposes darknet event detection methodologies to investigate the extent to which challenges we encountered extend to method assessment at large.
% Section 1 first provides an overview of canonical literature that characterizes darknet traffic and discusses major findings of previous studies. 
% Section 2 introduces our taxonomy of detection methodologies that breaks down their components and highlights the intuitions underlying each of their approaches.
% Section 3 reviews the empirical assessments of methods, identifying gaps that currently obscure a complete picture of the comparative capabilities between methods.
% We conclude our survey's findings with directions for future work aimed at enabling more rigorous, systematic assessments.

% In this report, we survey an extensive body of literature that proposes darknet event detection methodologies to investigate the extent to which challenges we encountered likewise extend to method assessment at large. 
% We begin with an overview of canonical literature that \dots in Section~\ref{sec:bg} .
% We then describe our taxonomy of methodologies in Section~\ref{sec:frameworks}, blah blah intution.
% In Section~\ref{sec:assessments}, we review their empirical assessments to identify gaps that prevent \dots.
% Finally, Section~\ref{sec:fw} closes our report with a discussion of future work we intend in order to enable more rigorous, systematic assessments of darknet event detection methodologies.

% In this report, we survey an extensive body of literature on darknet event detection methodologies to investigate the extent that the challenges we encountered affect this topic of research at large. 
% We begin with an overview of prior work that characterizes network telescope traffic and chronicles its major changes in Section~\ref{sec:bg}.
% We then describe our taxonomy of methodologies in Section~\ref{sec:frameworks}, comparing\dots\maxnote{brief summary}, followed by a review of their evaluation in Section~\ref{sec:assessments}.
% Finally, using our survey's findings, we conclude with a discussion of challenges in this area as well as directions for future work in Section~\ref{sec:fw}.

% In this report, we survey an extensive body of literature to construct a taxonomy of darknet event detection methodologies followed by a separate taxonomy that details their assessments in empirical evaluations. 
% We begin with an overview of prior work that characterizes network telescope traffic and chronicles its major changes in Section~\ref{sec:bg}.
% We then describe our taxonomy of methodologies in Section~\ref{sec:frameworks}, comparing\dots\maxnote{brief summary}, followed by a review of their evaluation in Section~\ref{sec:assessments}.
% Finally, using our survey's findings, we conclude with a discussion of challenges in this area as well as directions for future work in Section~\ref{sec:fw}.

% We begin this report with an overview of network telescopes, the traffic they collect, and the types of Internet-wide events inferrable through analysis of such traffic.
% We then survey various frameworks that have been proposed to automate event detection and construct a taxonomy that compares their 1) analysis goals and designs; and 2) performance as informed by empirical evaluations on real-world darknet traffic.
% Using this taxonomy, we identify challenges and opportunities for future research where efforts may yield improvements over the current state-of-art.

% \clearpage

\label{sec:bg}
\section{Background}
In this section, we provide a brief overview of network telescopes and discuss their role in Internet measurement research.
We first provide a technical explanation of their function as observatories of unsolicited IPv4 Internet traffic.
We then summarize findings from past works that have studied the nature and composition of this traffic during landmark Internet-wide events.
Finally, we remark on changes in characteristics of such traffic up to present day and speculate on the future development of its traffic and flows.

% \label{sec:bg:nt}
% \subsection{Network Telescopes}

\begin{figure*}[ht!]
    \centering
    \includegraphics[width=\textwidth]{figures/traffic_growth.pdf}
    \caption{Growth in volumes of darknet-traffic collected at UCSD-NT over 15 years. Daily compressed PCAP filesizes have exceeded 3.5 terabytes per day.}
    \label{fig:traffic_growth}
\end{figure*}

Network telescopes, or darknets, consist of IPv4 address space that receives, but does not respond to, unsolicited Internet traffic via routes announced through the Border Gateway Protocol (BGP).
Researchers have studied this unidirectional traffic to understand its mixtures and origins using darknets as its instrumentation with foundational work attributing its causes to Internet-wide activity that includes
malicious scanning~\cite{2024griffioen,2015dainotti,2017antonakakis,2023bischof,2014durumeric}, 
residual backscatter from DoS attacks~\cite{2014rossow,2017jonker,2017blenn,2021griffioen,2023nawrocki,2024hiesgen}, 
Internet outages~\cite{2011dainotti,2013benson,2012dainotti,2021padmanabhan}, 
and non-trivial network misconfigurations~\cite{2015benson}.

Large-scale empirical studies have characterized darknet-traffic composition and its changes over the past decades. 
In 2004, Pang et al.~\cite{2004pang} was the first to conduct a systematic analysis of darknet traffic\footnote{Pang et al. used a /8, two /19, and ten /24 sized darknets.} in terms of its activities and source, 
followed by a study in 2010 by Wustrow et al.~\cite{2010wustrow} that found traffic volumes\footnote{Wustrow et al. used five different /8 darknets.} 
had grown nearly four-fold along with a reversal in SYN/SYN-ACK trends in the years since. 
Later in 2014, following the release of high-speed scanning tools (ZMap~\cite{2013durumeric} and Masscan~\cite{masscan}), 
Durumeric et al.~\cite{2014durumeric} report\footnote{Durumeric et al. used a darknet roughly the size of a /9.} that horizontal-scans have become common with most malicious scans originating from bullet-proof hosting providers. 
More recently in 2024, Griffioen et al.~\cite{2024griffioen} conduct a 10-year longitudinal study and find a 30-fold increase, which roughly mirrors the trend pictured in Figure~\ref{fig:traffic_growth} for UCSD-NT, 
in scan traffic\footnote{Griffioen et al. use their /16 darknet to draw comparisons.} whose sources rapidly change geographic locations and port targets.
Parallel to this line of work, IPv4 address space exhaustion pressures have led researchers to study darknet traffic collected at non-traditional vantage points 
at IXPs~\cite{2023wagner}, CDNs~\cite{2019richter}, and cloud infrastructure~\cite{2023pauley} though the characteristics of this traffic remains to be extensively compared against traditional darknet traffic.

Despite the evolution of darknet traffic and instrumentation for its collection, researchers continue to devise methods and empirically demonstrate their effectiveness at detecting the various kinds of Internet-wide activities known to occur.
Our survey reviews the design, implementation, and assessments of these methods with an emphasis on those motivated by recent interest in the application of modern machine-learning based approaches to analyze darknet traffic.
% Early studies focused on characterizing the composition of traffic and often uncover patterns that inform detection approaches.
% % Yegneswaran et al.~\cite{2004yegneswaran} quantified the volume of traffic and distributions of senders responsible for originating malicious scans, 
% % subsequently constructing estimates that 25 billion Internet-wide scans originate per day.
% Bellovin~\cite{1993bellovin} earliest published report on the type of packets received by their unused address spaces.
% Pang et al.~\cite{2004pang} later offered the first systematic characterization of darknet traffic composition, including protocol- and application-level breakdowns and analyses of packet content.
% Wustrow et al.~\cite{2010wustrow} provide a follow-up characterization six years later, showed that while absolute traffic volumes had grown substantially, several structural properties—such as port distributions and temporal patterns—remained stable over time.
% Later efforts direct their analysis to specific phenomena: 
% Casado et al.~\cite{2005casado} examined the network-level behavior of scanning worms; 
% others used darknets as passive vantage points for identifying internet outages and routing instabilities~\cite{2011dainotti,2013benson,2015benson,2012dainotti,2021padmanabhan}.

\label{sec:methods}
% \section{Methodologies for detecting darknet scanning events}
\section{Detection methods for darknet scanning activities}

In this section, we provide a broad survey of the methods that have been proposed to detect scanning activity in darknet traffic. 
We construct a body of 35 works published over the past 14 years (summarized in Table~\ref{tab:methodologies} 
which consists of a majority that appear after the official release of ZMap in 2013~\cite{2013durumeric}.
Our single selection criterion includes only works that provided an assessment of their 
proposed method using real-world darknet traffic data\footnote{We made an exception for ~\cite{2020cohen} 
which used greynet traffic in their assessment.}
We organize these works by the key technique(s) their detection methods employ, describe major 
components of their designs, and discuss their strengths and weaknesses based on empirical findings where possible.
We briefly define these components before elaborating on individual works.

% In this section, we provide a broad survey of the methods that have been proposed to detect scanning activity in darknet traffic. 
% To construct the body of works for this survey, we selected only those that assessed their method on real-world traffic.
% We review 20 works from the past 10 years (summarized in Table~\ref{tab:methodologies}), organized based on the key technique(s) their detection methods employ.
% In addition, we describe several additional components of each method's design that we briefly discuss before elaborating on individual works.

\begin{comment}
% A portion of darknet traffic mixtures results from scans that probe the space of Internet addresses, oftentimes targeting application-layer ports.
% Detection of these scans involves identifying their sources and targets through analysis of behavioral patterns, thereby enabling further inference and/or classification of their intent.
% This section provides a survey of the methodologies proposed for detecting scanning activities in darknet traffic. 
% In this section, we provide a broad survey of the methods that have been proposed for the detection of scanning activity from darknet traffic. 
% We review 20 works from the past 10 years (summarized in Table~\ref{tab:methodologies}) and organize their proposed methods based on general detection task they perform.
% We briefly discuss components of each methodology's design, which include: features of darknet traffic they ingest, traffic representations these features are transformed into, and the general techniques and specific algorithms that operate on such traffic representations to achieve their detection task.
\end{comment}

\begin{comment}
% In this section, we provide a broad survey of existing methodologies that have been proposed for detecting darknet activities and 
% organize prior works into a taxonomy that identifies the need for a systematic evaluation approach to account for the diversity of methods.
% We structure our survey by categorizing detection methodologies primarily based on the type of darknet activity they aim to detect. 
% Within these categories, we review and emphasize two important aspects of each methodology: 
% 1) the class(es) of general \textit{techniques} they employ for detection; and 
% 2) their definition(s) of an \textit{event} belonging under their broader type of targeted activity.
% We briefly discuss these aspects before discussing the methodologies that comprise our taxonomy as shown in Figure~\ref{fig:taxonomy}.

% Our survey structure consists of a top-level categorization of detection methodologies based on the broad class of (i) darknet activity they aim to detect.
% Under these categories, we group similar methods by the (ii) general classes of techniques they employ to achieve their (iii) detection objectives.
% Using these groupings, we summarize the intuition behind each method and elaborate on finer-grained details pertaining to their specific (iv) algorithms and (v) traffic features.
% Here, we briefly discuss components of the methods we survey.

% We structure our survey by categorizing detection methodologies primarily based on the type of darknet activity they aim to detect. 
% Within these categories, we review and emphasize two important aspects of each methodology: 
% 1) the class(es) of general \textit{techniques} they employ for detection; and 
% 2) their definition(s) of an \textit{event} belonging under their broader type of targeted activity.
% We briefly discuss these aspects before discussing the methodologies that comprise our taxonomy as shown in Figure~\ref{fig:taxonomy}.
\end{comment}

\begin{table}[t]
    \small
    \centering
    \caption{Summary of our surveyed detection methods and their components.}
    \label{tab:methodologies}
    \renewcommand{\arraystretch}{1.1}
    \setlength{\tabcolsep}{3.5pt}
    \begin{tabular}{c l ccc ccccccc p{3cm} cccc}
    \toprule
    &
    \textbf{Work} &
    \multicolumn{3}{@{}c@{}}{\begin{tabular}[c]{@{}c@{}}\textbf{Detection}\\\textbf{Tasks}\end{tabular}} &
    \multicolumn{7}{@{}c@{}}{\textbf{Techniques}} &
    \multicolumn{1}{@{}c@{}}{\begin{tabular}[c]{@{}c@{}}\textbf{Traffic}\\\textbf{Features}\end{tabular}} &
    \multicolumn{4}{@{}c@{}}{\begin{tabular}[c]{@{}c@{}}\textbf{Traffic Repre-}\\\textbf{sentations}\end{tabular}} \\
    \cmidrule(lr){3-5}\cmidrule(lr){6-12}\cmidrule(lr){13-13}\cmidrule(lr){14-17}

    % \multirow{-2}{*}{\rot{90}{Group}}
    \rotatebox{90}{\textbf{Group}}
    % \textbf{Group}
    &
    & \rot{90}{Source Class.} & \rot{90}{Target Id.} & \rot{90}{Event Det.} &
    \rot{90}{Clustering} & \rot{90}{Dim.\ Reduction} & \rot{90}{Forecasting} &
    \rot{90}{Thresholding} & \rot{90}{Fingerprinting} & \rot{90}{Rep.\ Learning} &
    \rot{90}{Pattern Mining} &
    &
    \rot{90}{Graph} & \rot{90}{Feature Vector} & \rot{90}{Sequences} & \rot{90}{Time Series} \\
    \midrule

    % =====================
    \multirow{5}{*}{\rotatebox{90}{Rep.\ learning}} &
        2025 Abduaziz et al.~\cite{2025abduaziz}
        & \cmark & &
        & \cmark & \cmark & & & & \cmark &
        & Per-\textit{saddr}: \textit{seq(dport)}
        & & & \cmark & \\
    \cline{2-17}

    &   2024 Huang et al.~\cite{2024huang}
        & \cmark & & \cmark
        & \cmark & \cmark & & & & \cmark &
        & Per-\textit{dport}: \textit{seq(saddr)}
        & & & & \cmark \\
    \cline{2-17}

    &   2022 Kallitsis et al.~\cite{2022kallitsis}
        & \cmark & & \cmark
        & \cmark & & & & & \cmark &
        & Per-\textit{saddr}: \textit{12 features}
        & & \cmark & & \\
    \cline{2-17}

    &   2021 Gioacchini et al.~\cite{2021gioacchini}
        & \cmark & &
        & \cmark & & & & & \cmark &
        & Per-\textit{dport}: \textit{seq(saddr)}
        & & & \cmark & \\
    \cline{2-17}

    &   2020 Cohen et al.~\cite{2020cohen}
        & & \cmark & \cmark
        & \cmark & & & & & \cmark &
        & Per-\textit{saddr}: \textit{dport}
        & & & & \cmark \\
    \midrule[0.1em]

    % =====================
    \multirow{5}{*}{\rotatebox{90}{Graph-mining}} &
        2023 Zakroum et al.~\cite{2023zakroum}
        & \cmark & &
        & \cmark & & & & & \cmark &
        & Per-\textit{saddr}: \textit{28+ features}
        & \cmark & & & \\
    \cline{2-17}

    &   2023 d’Andréa et al.~\cite{2023dandrea}
        & & \cmark & \cmark
        & & & & & & \cmark &
        & Per-\textit{saddr}: \textit{seq(dport)}
        & \cmark & & & \\
    \cline{2-17}

    &   2020 Soro et al.~\cite{2020soro}
        & \cmark & &
        & \cmark & & & & & &
        & Per-(\textit{AS,dport}): \textit{cnt(pkt)}
        & & \cmark & & \\
    \cline{2-17}

    &   2019 Evrard et al.~\cite{2019evrard}
        & \cmark & &
        & \cmark & & & & & &
        & Per-\textit{saddr}: \textit{seq(dport)}
        & \cmark & & & \\
    \cline{2-17}

    &   2017 Lagraa et al.~\cite{2017lagraa}
        & \cmark & &
        & \cmark & & & & & &
        & Per-\textit{saddr}: \textit{seq(dport)}
        & \cmark & & & \\
    \midrule[0.1em]

    % =====================
    \multirow{8}{*}{\rotatebox{90}{Time series}} &
        2024 Gao et al.~\cite{2024gao}
        & \cmark & \cmark & \cmark
        & & & & \cmark & & & \cmark
        & \makecell[l]{
            Per-\textit{dport}: \textit{uniq(saddr)} \\
            Per-\textit{country}: \textit{cnt(pkt)}
        }
        & & & & \cmark \\
    \cline{2-17}

    &   2023 Kartsioukas et al.~\cite{2023kartsioukas}
        & & \cmark & \cmark
        & & \cmark & & \cmark & & &
        & Per-\textit{dport}: \textit{uniq(saddr)}
        & & & & \cmark \\
    \cline{2-17}

    &   2022 Zakroum et al.~\cite{2022zakroum}
        & & \cmark & \cmark
        & \cmark & & \cmark & & & &
        & Per-\textit{dport}: \textit{cnt(pkt)}
        & & & & \cmark \\
    \cline{2-17}

    &   2021 Han et al.~\cite{2021han}
        & \cmark & \cmark & \cmark
        & & \cmark & & \cmark & & &
        & \makecell[l]{
            Per-\textit{saddr}: \textit{cnt(pkt)} \\
            Per-\textit{saddr16}: \textit{cnt(pkt)}
        }
        & & & & \cmark \\
    \cline{2-17}

    &   2020 Han et al.~\cite{2020han}
        & \cmark & \cmark & \cmark
        & & \cmark & & & & &
        & Per-\textit{saddr}: \textit{cnt(pkt)}
        & & & & \cmark \\
    \cline{2-17}

    &   2019 Kanehara et al.~\cite{2019kanehara}
        & \cmark & \cmark & \cmark
        & & \cmark & & \cmark & & &
        & Per-(\textit{saddr,dport}): \textit{cnt(pkt)}
        & & & & \cmark \\
    \cline{2-17}

    &   2017 Ban et al.~\cite{2017ban}
        & \cmark & \cmark & \cmark
        & & & & \cmark & & &
        & \makecell[l]{
            Per-\textit{dport}: \textit{uniq(saddr)} \\
            Per-\textit{saddr}: \textit{2 features}
        }
        & & & & \cmark \\
    \cline{2-17}

    &   2016 Ban et al.~\cite{2016ban}
        & \cmark & \cmark & \cmark
        & \cmark & & & & & & \cmark
        & \makecell[l]{
            Per-\textit{dport}: \textit{uniq(saddr)} \\
            Per-\textit{saddr}: \textit{2 features}
        }
        & & & & \cmark \\
    \midrule[0.1em]

    % =====================
    \multirow{2}{*}{\rotatebox{90}{FP}} &
        2021 Tanaka et al.~\cite{2021tanaka}
        & \cmark & &
        & & & & & \cmark & &
        & Per-\textit{saddr}: \textit{5 features}
        & & & & \\
    \cline{2-17}

    &   2020 Griffioen et al.~\cite{2020griffioen}
        & \cmark & &
        & & & & & \cmark & &
        & Per-\textit{saddr}: \textit{7 features}
        & & & & \\
    \midrule[0.1em]

    % =====================
    \multirow{4}{*}{\rotatebox{90}{N/A}} &
        2020 Torabi et al.~\cite{2020torabi}
        & & & \cmark
        & & & & & & & \cmark
        & Per-\textit{flow}: \textit{6 features}
        & \cmark & \cmark & & \\
    \cline{2-17}

    &   2019 Iglesias et al.~\cite{2019iglesias}
        & \cmark & & \cmark
        & \cmark & & & & & &
        & Per-\textit{saddr}: \textit{22 features}
        & & \cmark & & \\
    \cline{2-17}

    &   2019 Niranjana et al.~\cite{2019niranjana}
        & \cmark & &
        & \cmark & & & & & &
        & Per-\textit{saddr}: \textit{22 features}
        & & \cmark & & \\
    \cline{2-17}

    &   2015 Nishikaze et al.~\cite{2015nishikaze}
        & \cmark & &
        & \cmark & & & & & &
        & Per-\textit{saddr16}: \textit{5 features}
        & & \cmark & & \\

    % 2012 Fachka et al.~\cite{2012fachka}
    % & \cmark & & 
    % & & & & & & & \cmark
    % & saddr16: \textit{pktcnt, sport, dport, daddr, scantype}
    % & & & & \\
    \bottomrule
\end{tabular}
\end{table}

\vspace{0.5em}
\noindent{\textbf{Detection tasks.}}
Since each work uses different terminology to describe detection tasks of their method, we redefine 3 types of tasks in order to generalize their definitions.
\textit{Source characterization} refers to method-specific tasks that range from clustering/classification of darknet traffic sources and detection of potentially coordinated sources.
\textit{Target identification} involves identifying specific application-layer ports that anomalous traffic targets which may correspond to reconaissance activity and scans originated from malware-infected hosts.
\textit{Event detection} determines specific points in time during which dynamics of darknet traffic exhibit significant change. For example, ~\cite{2024huang,2022kallitsis} detect changes in sender behavior by tracking cluster changes over time.

% \vspace{0.25em}
% \noindent{\textbf{Class(es) of activity}} correspond to the canonical activities observed in darknet traffic as described in Section~\ref{sec:bg}. 
% For each study, we identify which class(es) they target based on explicitly stated analysis goals of each methodology and the empirical findings reported from their assessments.
% Some methods such as~\cite{2022kallitsis,2019iglesias} are capable of targeting multiple types of activities since traffic dynamics associated with scanning, backscatter, and misconfiguration can overlap. 
% In practice, studies such as~\cite{2019evrard} filter traffic inputs using heuristics (e.g., flags indicating TCP response traffic~\cite{2006moore}, likely-spoofed sender IP addresses~\cite{2015dainotti}) to ensure their methodology analyzes a specific type of activity.

% \vspace{0.25em}
% \noindent{\textbf{Event Definitions}} refers to attributes represented in the outputs of a given framework.
% These attributes derive from packet header fields and implicitly define an instance of a broader type of event.
% For example, a framework may define a scanning event using a source-based schema that consists of a timestamp, source IP address, an assigned cluster, and a cluster flag indicating potential malicious intent.

\vspace{0.25em}
\noindent{\textbf{Technique(s)}} differentiate functionalities undertaken by specific \textbf{algorithms(s)} within the scope of each methodology. 
Across our surveyed works, we identified 7 classes of techniques which include 
\textit{dimensionality reduction},
\textit{clustering},
\textit{forecasting},
\textit{thresholding},
\textit{representation learning},
\textit{frequent pattern mining},
and \textit{fingerprinting}.
Most methodologies employ a combination of techniques 
(\textit{e.g.}, dimensionality reduction as a prior step to clustering or thresholding similarity metrics computed from intermediary representations)
to accomplish analysis subtasks.

\vspace{0.25em}
\noindent{\textbf{Traffic features and representations.}} 
Methods consist of a preprocessing step that parses properties of raw darknet traffic into representations usable by their algorithms.
Depending on the method's detection task, traffic features are defined per-source, per-target, or per-flow at different levels of aggregation
and encoded as one of the four main types of representations: \textit{graphs}, \textit{feature vectors}, \textit{sequences}, \textit{time series}.

\subsection{Representation learning methods}
Representation learning techniques play a key role in modern machine-learning for their capability 
to automatically learn lower-dimensional representations of raw data for downstream modeling tasks.
While their functionality overlaps with classic dimensionality reduction techniques, 
for our survey we distinguish these methods by their application of artificial neural network architectures that learn 
non-linear relationships between features of darknet traffic.

Researchers first devised methodologies that incorporated these techniques into darknet traffic analysis starting in 2020 
with the Word2Vec~\cite{2013mikolov} algorithm.
Cohen et al.~\cite{2020cohen} proposed \textit{Dante} to detect scanning trends in TCP ports from clustered 
Word2Vec embeddings trained from \textit{port sequences} per sender.
Gioacchini et al.~\cite{2021gioacchini} demonstrated that using a modified embedding definition, 
\textit{i.e.,} \textit{sender sequences} per port, \textit{DarkVec} results in better scalability 
and more accurate identification of labeled scanner organizations compared to \textit{Dante}.
Follow-up works~\cite{2023gioacchini,2024huang} extend \textit{DarkVec} to handle larger volumes of traffic and track changes in clusters across time.
Abduaziz et al.~\cite{2025abduaziz} replicate \textit{Dante} though with a modified semi-supervised clustering technique to highlight 
In 2022, Kallitsis et al.~\cite{2022kallitsis} demonstrated that their method which employed deep-learning,
\textit{i.e.,} by applying autoencoders to embed 12 features selected to represent senders,
could more accurately cluster the same labeled senders as \textit{DarkVec} though at a cost of greater implementation complexity 
and computational requirements.

\subsection{Graph-based methods}
These methods represent the behaviors of darknet traffic sources and the 
sequence of their probed targets using different graph formulations.
While they leverage the same fundamental features as early Word2Vec methods, 
graph-mining methods in principle have lower computational requirements depending on the formulation used in analysis.

Methods proposed by Lagraa et al.~\cite{2017lagraa} and Evrard et al.~\cite{2019evrard}
represent sequences of scanned TCP ports as unipartite graphs (ports as nodes and consecutive scans as edges).
To identify groups of similarly probed ports, both methods threshold graph metrics 
(shortest-path similarity in ~\cite{2019evrard}, node centrality in ~\cite{2017lagraa}).
In follow-up work, Lagraa et al.~\cite{2019lagraa} provide an additional graph definition to detect horizontal darknet scanners.
Soro et al.~\cite{2020soro} propose modeling interactions as weighted bipartite graphs which 
enables a wider scope of analysis but at the cost scaling limitations which led to mapping sender IPs to 
a lower spatial grain, Autonomous Systems (ASes).
Application of the Greedy Modularity Algorithm resulted in different clusters of ASes that separately contained
distributed scanners targeting specific ports, horizontal scanners, and potential misconfiguration-related traffic.

More recent work incorporates representation learning techniques to graph representations of darknet traffic.
Using the same graph formulation as ~\cite{2019evrard},
d’Andréa et al.~\cite{2023dandrea} apply a Graph Convolutional Network (GCN) to classify senders labeled using AbuseIPDB reports.
Zakroum et al.~\cite{2023zakroum} learn graph-embeddings that model packet-level changes in flows.

% Zakroum et al.~\cite{2023zakroum} 
% ~\cite{2022zakroum}

% Gioacchini et al.~\cite{2021gioacchini,2023gioacchini} propose a domain-specific adaptation of Word2Vec to embed senders based on their order of arrival from packet sequences \maxnote{defined over variations of port combinations}.
% Kallitsis et al.~\cite{2022kallitsis} include a more comprehensive set of features to train autoencoders and subsequently cluster sender embeddings of a lower-dimensional space. 
% Nishikaze et al.~\cite{2015nishikaze} cluster feature vectors that represent source /16 subnets and 27 selected features.
% Zakroum et al.~\cite{2023zakroum} 

% \vspace{0.25em}
% \noindent{\textbf{Time Series Dimensionality Reduction Methods}}
\subsection{Time series-based methods}
Time series offer the flexibilty to represent a variety of traffic metrics that together offer an aggregate view of darknet activity.
Methods based on these representations typically analyze a large number of time series as their inputs with the goal of 
identifying sources or targeted ports of anomalous traffic activity.

Several methods apply dimensionality reduction techniques, using linear decomposition algorithms which are able to 
practically separate anomalies from components of noisy darknet traffic.
Kanehara et al.~\cite{2019kanehara} applied Nonnegative Tucker Decomposition (NTD)~\cite{2015zhou} 
to packet count time series for pairs of senders and destination ports.
Han et al.~\cite{2020han} proposed \textit{DarkNMF} and \textit{DarkGLASSO}, which respectively applied Nonnegative Matrix Factorization (NMF)~\cite{2000lee} and 
Graphical LASSO~\cite{2008friedman} to packet count time series per source /24 network (for \textit{DarkNMF}, also per destination port).
Later comparisons of the three methods in ~\cite{2022han} found that of the two matrix factorization methods, 
NMF required less computation but produced greater number of false positives; of all three, 
GLASSO was the most expensive to compute and offered similar detection performance to NTD.
Separate from those works, Kartsioukas et al.~\cite{2023kartsioukas} apply 
Incremental Principal Component Analysis (iPCA)~\cite{2012arora} 
to time series of unique senders counts arriving on TCP, UDP ports and ICMP traffic.

In contrast to the previous works, the remaining methods draw on a range of techniques with less overlap.
Zakroum et al.~\cite{2022zakroum} detect abnormal traffic volumes on ports that deviate from their forecasted rates and correlate 
them against public vulnerability disclosures.
Our recent proposed method~\cite{2024gao} uses time series similarity metrics to enable pattern template-matching and anomalous event detection.
Ban et al.~\cite{2016ban} cluster time series of unique sender counts arriving on ports grouped by an upstream frequent itemset algorithm.
To identify coordinated sources, 
Ban et al.~\cite{2017ban} propose a two-part method that
first detects bursts of senders per port using 
adaptive thresholds and then classifies associated senders.

\subsection{Fingerprinting methods}
Fingerprinting methods operate on information embedded in packet headers of darknet traffic. 
The key intuition behind the use of these methods are that distributed yet coordinated hosts likely employ the same 
stateless scanning tools, \textit{e.g.}, ZMap~\cite{2013durumeric}, embed identifying information in 
the same parts of a packet which can be used to infer their fingerprints.
Griffioen et al.~\cite{2020griffioen} first proposed this technique which enabled correlation of similar hosts suspected 
to use the same scanning tools.
Tanaka et al.~\cite{2021tanaka,2023tanaka} extend Griffioen et al.'s method to generate candidate fingerprints from a 
larger set of header fields using a genetic algorithm.

% \subsection{Pattern-mining methods}

% Nishikaze et al.~\cite{2015nishikaze}
% Torabi et al.~\cite{2020torabi}

% \label{sec:methods:dos}
% \subsection{Backscatter Detection}
% Several of the previously mentioned methodologies in Section~\ref{sec:methods:scan} detect both scans and backscatter since reflected response traffic generated by spoofed DoS attacks can resemble scanning behavior, e.g., a DoS victim's response to spoofed traffic compared to high-volume horizontal scans.
% While distinguishing backscatter from scans in stateful protocol traffic such as TCP, e.g., by inspecting untampered TCP flags, for stateless protocols such as ICMP and UDP it is less so. 
% \maxnote{enumerate works here}

% \label{sec:methods:outage}
% \subsection{Outage Detection}
% In contrast to scanning and backscatter detection, outage detection methodologies aim to identify the absence of expected traffic.
% These methodologies frame detection in terms of traffic sources defined at varying spatial grains (e.g., per-subnet, per-AS, or per-country) which may depend on darknet-exogenous data for geolocation or AS-organization lookups.

% Our survey identifies two works that propose complete outage detection methodologies.
% Guillot et al. propose Chocolatine~\cite{2019guillot} which forecasts the expected number of unique senders aggregated at per-AS and per-country granularities by training a Seasonal Autoregressive Integrated Moving Average (SARIMA) model over historical sender counts of unfiltered, protocol-agnostic traffic inputs; 
% sender counts beyond specified 5-minute prediction thresholds are flagged as potential outages.
% Enyanet et al. propose Durbin~\cite{2024enyanet} which leverages a Bayesian approach for inferring outages down to a coarser spatial grain of a /24 subnetwork in comparison to Chocolatine. Furthermore, Durbin optimizes parameters per-subnet and and claims to more flexibly exploit space-time precision trade-offs by doing so. 

\label{sec:assessments}
% \section{Empirical Assessments of Detection Methodologies}
\section{Assessments of detection methodologies}

\begin{table*}[t!]
    \small
    \setlength{\tabcolsep}{1pt}
    \caption{
        Details of method assessments found in our surveyed works. 
        Multiple citations per entry indicate groups of highly similar works; assessment details reflect bolded citations.
        Table~\ref{tab:telescopes} lists additional details of telescopes referenced in this table.
    }
    \label{tab:eval}
    \begin{tabular}{@{}c @{\hspace{3.5pt}}lcccccccccccc@{}}
        \toprule
        & \textbf{Work}
        & \multicolumn{3}{c}{\bf Replicability} 
        & \multicolumn{5}{c}{\bf Dataset Attributes} 
        & \multicolumn{3}{c}{\textbf{Validation}} \\
        \cmidrule(lr){3-5} \cmidrule(lr){6-10} \cmidrule(lr){11-13}
        \addlinespace[0.5em]
        \multirow{-2.5}{*}{\rotatebox{90}{\textbf{Group}}} 
        &
        & \textbf{Code} & \textbf{Specs} & \textbf{Data}
        & \textbf{Source(s)} & \textbf{Duration(s)} & \textbf{Year(s)} & \textbf{Packets} & \textbf{Bytes} 
        & \textbf{Comp.} & \textbf{Labels} & \textbf{Ext.} \\
        \midrule

        % =====================
        \multirow{5}{*}{\rotatebox{90}{Rep.\ learning}} &
        Abduaziz et al.~\cite{2025abduaziz}
        & --- & \cmark & ---
        & NT-1,3
        & 7,10D & 2022,23
        & --- & 1.18, 7.61GB
        & \cite{2020cohen}
        & \cmark
        & \cmark \\
        \cline{2-13}

        & Huang et al.~\cite{2024huang}
        & --- & --- & ---
        & NT-4
        & 50D & 2021
        & --- & ---
        & ---
        & \cmark
        & \cmark \\
        \cline{2-13}

        & Kallitsis et al.~\cite{2022kallitsis}
        & \cmark & \cmark &  ---
        & NT-2
        & 28,1D & 2016,22
        & 49B, 3.1B & ---
        & \cite{2021gioacchini}
        & \cmark
        & \cmark \\
        \cline{2-13}

        & Gioacchini et al.\textbf{~\cite{2021gioacchini}}~\cite{2023gioacchini}
        & \cmark & \cmark & \cmark
        & NT-4
        & 30D & 2021
        & 63M & ---
        & \cite{2020cohen,2017ring}
        & \cmark
        & \cmark \\
        \cline{2-13}
        
        & Cohen et al.~\cite{2020cohen}
        & --- & \cmark & ---
        & GT-1
        & 44,55,257D & 2018,19,19
        & 7.9B & 3+TB
        & \cite{2016ban}
        & ---
        & --- \\
        \midrule

        % ===================== 
        \multirow{5}{*}{\rotatebox{90}{Graph-mining}} &
        d’Andréa et al.~\cite{2023dandrea}
        & --- & \cmark & ---
        & NT-6
        & 30D & 2021
        & $\sim$195M & ---
        & ---
        & \cmark
        & \cmark \\
        \cline{2-13}

        & Zakroum et al.~\cite{2023zakroum}
        & --- & --- & ---
        & NT-3,6
        & 4.5,4.5Y & 2018,18
        & --- & ---
        & \cite{2023zakroum}
        & \cmark
        & \cmark \\
        \cline{2-13}

        & Soro et al.~\cite{2020soro}
        & --- & --- & ---
        & NT-4,5
        & 3W,1D & 2020
        & --- & ---
        & ---
        & ---
        & --- \\
        \cline{2-13}

        & Evrard et al.~\cite{2019evrard}
        & --- & --- & ---
        & NT-3,6
        & 9,9M & 2015,15
        & $\sim$815M & ---
        & ---
        & ---
        & \cmark \\
        \cline{2-13}

        & Lagraa et al.\textbf{~\cite{2019lagraa}}~\cite{2017lagraa}
        & --- & \cmark & ---
        & NT-6
        & 2Y & 2014
        & 2.8B & 500 GB
        & ---
        & ---
        & --- \\
        \midrule

        % =====================
        \multirow{8}{*}{\rotatebox{90}{Time series}} &
        Gao et al.~\cite{2024gao}
        & \cmark & \cmark & --- 
        & NT-1
        & 18M & 2022
        & --- & ---
        & \cite{2020han}
        & ---
        & --- \\
        \cline{2-13}

        & Kartsioukas et al.~\cite{2023kartsioukas}
        & --- & --- & ---
        & NT-2
        & 1M & 2016
        & --- & ---
        & \cite{2004lakhina}
        & ---
        & --- \\
        \cline{2-13}

        & Zakroum et al.\textbf{~\cite{2022zakroum}}~\cite{2018zakroum}
        & --- & \cmark & ---
        & NT-3,6
        & 3.5,3.5Y & 2017,17
        & --- & 1.5+TB
        & \cite{2018zakroum,2022zakroum}
        & ---
        & \cmark \\
        \cline{2-13}

        & Han et al.~\cite{2022han}
        & \cmark & \cmark & \cmark
        & NT-3
        & 1M & 2018
        & --- & ---
        & \cite{2020han,2006takeuchi,2019kanehara}
        & \cmark
        & --- \\
        \cline{2-13}

        & Han et al.\textbf{~\cite{2021han}}~\cite{2022han}
        & --- & --- & \cmark
        & NT-3
        & 1M & 2018
        & --- & ---
        & \cite{2020han,2006takeuchi}
        & \cmark
        & --- \\
        \cline{2-13}

        & Han et al.\textbf{~\cite{2020han}}~\cite{2022han}
        & --- & \cmark & \cmark
        & NT-3
        & 1M & 2018
        & --- & ---
        & \cite{2006takeuchi}
        & \cmark
        & --- \\
        \cline{2-13}

        & Kanehara et al.\textbf{~\cite{2019kanehara}}~\cite{2022han}
        & --- & \cmark & ---
        & NT-3
        & 7M & 2018
        & --- & ---
        & ---
        & ---
        & --- \\
        \cline{2-13}

        & Ban et al.~\cite{2017ban}
        & --- & --- & ---
        & NT-3
        & 1Y & 2015
        & --- & ---
        & \cite{2012ban}
        & \cmark
        & --- \\
        \cline{2-13}

        & Ban et al.~\cite{2016ban}
        & --- & ---  & ---
        & NT-3
        & 1Y & 2015
        & --- & ---
        & ---
        & ---
        & --- \\
        \midrule
        % =====================
        \multirow{2}{*}{\rotatebox{90}{FP}} &
        Tanaka et al.\textbf{~\cite{2023tanaka}}~\cite{2021tanaka}
        & --- & \cmark & ---
        & NT-3
        & 1M & 2021
        & 37B & ---
        & ---
        & \cmark
        & \cmark \\
        \cline{2-13}

        & Griffioen et al.~\cite{2020griffioen}
        & --- & --- & ---
        & NT-8
        & 2M & 2018
        & 6.5B & 864G
        & ~\cite{2020griffioen}
        & ---
        & --- \\
        \midrule

        % =====================
        \multirow{7}{*}{\rotatebox{90}{N/A}} &
        Torabi et al.~\cite{2020torabi}
        & --- & --- & ---
        & NT-1
        & 5,6D & 2017,18
        & --- & 6+TB
        & ---
        & \cmark
        & \cmark \\
        \cline{2-13}

        & Iglesias et al.~\cite{2019iglesias}
        & --- & \cmark & \cmark
        & NT-1
        & 6M & 2012
        & --- & 100TB
        & ---
        & ---
        & --- \\
        \cline{2-13}

        & Niranjana et al.~\cite{2019niranjana}
        & --- & --- & ---
        & Unk. /24
        & 20D & 2017
        & --- & ---
        & ---
        & ---
        & --- \\
        \cline{2-13}

        & Bou-Harb et al.\textbf{~\cite{2019bouharb}}~\cite{2015bouharb}
        & --- & \cmark & ---
        & NT-1,7
        & 1,1M & 2016,14
        & --- & 670, 240GB
        & ~\cite{2018bouharb,2011li}
        & \cmark
        & \cmark \\
        \cline{2-13}

        & Nishikaze et al.~\cite{2015nishikaze}
        & --- & --- & ---
        & NT-3
        & 28D & 2014
        & 303M & ---
        & ---
        & ---
        & --- \\
        \cline{2-13}

        % & Bou-Harb et al.~\cite{2014bouharb}
        % &  &  & ---
        % & NT-7
        % & 2D & 2013
        % & $10^6$ & 30GB
        % & 
        % & 
        % &  \\
        % \midrule

        % & Fachka et al.~\cite{2012fachka}
        % &  &  & 
        % & NT-7
        % & 2D & 2011
        % & --- & ---
        % & 
        % & 
        % &  \\

        & \textbf{Aggregate}
        & 4/26 & 14/26 & ---
        & ---
        & 1D--3.5Y & 2012--2023
        & --- & ---
        & 14/26
        & 13/26
        & 11/26 \\
        \bottomrule
    \end{tabular}
\end{table*}
\begin{comment}
\begin{table*}[t!]
    \small
    \setlength{\tabcolsep}{1.5pt}
    \caption{
        Details of method assessments found in our surveyed works. 
        Multiple citations per entry indicate groups of highly similar works; assessment details reflect bolded citations.
        Table~\ref{tab:telescopes} lists additional details of telescopes referenced in this table.
    }
    \label{tab:eval}
    \begin{tabular}{@{}c lcccccccccc@{}}
        \toprule
        & \textbf{Work}
        & \multicolumn{3}{c}{\bf Replicability} 
        & \multicolumn{5}{c}{\bf Dataset Attributes} 
        & \multicolumn{1}{c}{\textbf{Compar-}}
        & \multicolumn{1}{c}{\textbf{Labels}} \\
        \cmidrule(lr){3-5} \cmidrule(lr){6-10}
        \addlinespace[0.5em]
        \multirow{-2.5}{*}{\rotatebox{90}{\textbf{Group}}} 
        &
        & \textbf{Code} & \textbf{Specs} & \textbf{Data}
        & \textbf{Telescope(s)} & \textbf{Duration} & \textbf{Year} & \textbf{Packets} & \textbf{Bytes} 
        & \textbf{ison} 
        & \\
        \midrule

        % =====================
        \multirow{5}{*}{\rotatebox{90}{Rep.\ learning}} &
        Abduaziz et al.~\cite{2025abduaziz}
        &  & \cmark & 
        & NT-1,3
        & 7D,10D & 2022,2023
        & --- & 1.18, 7.61GB
        & \cite{2020cohen}
        & \cmark \\
        \cline{2-12}

        & Huang et al.~\cite{2024huang}
        & & & 
        & NT-4
        & 50D & 2021
        & --- & ---
        & 
        & \cmark \\
        \cline{2-12}

        & Kallitsis et al.~\cite{2022kallitsis}
        & \cmark & \cmark & 
        & NT-2
        & 28D, 1D & 2016,2022
        & 49B, 3.1B & ---
        & \cite{2021gioacchini}
        & \cmark \\
        \cline{2-12}

        & Gioacchini et al.\textbf{~\cite{2021gioacchini}}~\cite{2023gioacchini}
        & \cmark & \cmark & \cmark
        & NT-4
        & 30D & 2021
        & 63M & ---
        & \cite{2020cohen,2017ring}
        & \cmark \\
        \cline{2-12}
        
        & Cohen et al.\textbf{~\cite{2020cohen}}
        & & \cmark &
        & NT-4
        & 1Y & 
        & 7.9B & 
        & \cite{2016ban}
        & \\
        \midrule

        % ===================== 
        \multirow{5}{*}{\rotatebox{90}{Graph-mining}} &
        d’Andréa et al.~\cite{2023dandrea}
        & & \cmark & 
        & Private /20
        & & 2021
        & --- & ---
        & 
        & \cmark \\
        \cline{2-12}

        & Zakroum et al.~\cite{2023zakroum}
        &  &  & 
        & NT-3,6
        & 4.5Y & 2018
        & --- & ---
        & \cite{2023zakroum}
        & \cmark \\
        \cline{2-12}

        & Soro et al.~\cite{2020soro}
        &  &  & 
        & NT-4,5
        & 3W,1D & 2020
        & --- & ---
        & 
        & \\
        \cline{2-12}

        & Evrard et al.~\cite{2019evrard}
        &  &  & ---
        & NT-3,6
        & 9M & 2015
        & 8M & ---
        & 
        & \\
        \cline{2-12}

        & Lagraa et al.\textbf{~\cite{2019lagraa}}~\cite{2017lagraa}
        &  & \cmark &
        & NT-6
        & 2Y & 2014
        & 2B & 500 GB
        & 
        & \\
        \midrule

        % =====================
        \multirow{8}{*}{\rotatebox{90}{Time series}} &
        Gao et al.~\cite{2024gao}
        & & & 
        & NT-1
        & 18M & 2022
        & --- & ---
        & \cite{2020han}
        & \\
        \cline{2-12}

        & Kartsioukas et al.~\cite{2023kartsioukas}
        &  &  & 
        & NT-2
        & 1M & 2016
        & --- & ---
        & \cite{2004lakhina}
        & \\
        \cline{2-12}

        & Zakroum et al.\textbf{~\cite{2022zakroum}}~\cite{2018zakroum}
        &  & \cmark & 
        & NT-3,6
        & 3.5Y & 2017
        & --- & $1.5+\mathrm{TB}$
        & \cite{2018zakroum}
        & \\
        \cline{2-12}

        & Han et al.~\cite{2022han}
        & \cmark & \cmark & \cmark
        & NT-3
        & 1M & 2018
        & --- & ---
        & \cite{2020han,2006takeuchi,2019kanehara}
        & \cmark \\
        \cline{2-12}

        & Han et al.\textbf{~\cite{2021han}}~\cite{2022han}
        & \cmark & \cmark & \cmark
        & NT-3
        & 1M & 2018
        & --- & ---
        & \cite{2020han,2006takeuchi,2019kanehara}
        & \cmark \\
        \cline{2-12}

        & Han et al.\textbf{~\cite{2020han}}~\cite{2022han}
        & \cmark & \cmark & \cmark
        & NT-3
        & 1M & 2018
        & --- & ---
        & \cite{2006takeuchi}
        & \cmark \\
        \cline{2-12}

        & Kanehara et al.\textbf{~\cite{2019kanehara}}~\cite{2022han}
        & & \cmark & 
        & NT-3
        & 10M & 2017
        & --- & ---
        & 
        & \\
        \cline{2-12}

        & Ban et al.~\cite{2017ban}
        &  &  & ---
        & NT-3
        & --- & ---
        & --- & ---
        & \cite{2012ban}
        & \\
        \cline{2-12}

        & Ban et al.~\cite{2016ban}
        &  &  & ---
        & NT-3
        & 1y & 2015
        & $3\times10^{7}$ & ---
        & 
        & \\
        \midrule
        % =====================
        \multirow{2}{*}{\rotatebox{90}{FP}} &
        Tanaka et al.~\cite{2021tanaka}
        & & \cmark & \cmark
        & NT-3
        & 1M & 2018
        & 117M & ---
        & \cite{2020han,2006takeuchi,2019kanehara}
        & \cmark \\
        \cline{2-12}

        & Griffioen et al.\textbf{~\cite{2020griffioen}}
        & & \cmark &
        & Private
        & 2M & 
        & 6.5B & 864G
        & \cite{2016ban}
        & \\
        \midrule

        % =====================
        \multirow{7}{*}{\rotatebox{90}{N/A}} &
        Torabi et al.~\cite{2020torabi}
        & \cmark & \cmark & ---
        & 
        & & 
        & --- & ---
        & 
        & \\
        \cline{2-12}

        & Iglesias et al.~\cite{2019iglesias}
        &  & \cmark & 
        & NT-1
        & 6M & 2012
        & --- & 2.1 TB
        & 
        & \\
        \cline{2-12}

        & Niranjana et al.~\cite{2019niranjana}
        &  &  & 
        & Private /24
        & 20D & 2017
        & & ---
        & 
        & \\
        \cline{2-12}

        & Bou-Harb et al.\textbf{~\cite{2019bouharb}}~\cite{2015bouharb}
        &  & \cmark & ---
        & NT-1,7
        & 1M,1M & 2016,2014
        & --- & 670, 240GB
        & \cite{2018bouharb}
        & \\
        \cline{2-12}

        & Nishikaze et al.~\cite{2015nishikaze}
        &  &  & 
        & NT-3
        & 28D & 2014
        & 303M & ---
        & 
        & \\
        \cline{2-12}

        & Bou-Harb et al.~\cite{2014bouharb}
        &  &  & ---
        & NT-7
        & 2D & 2013
        & $10^6$ & 30GB
        & 
        & \\
        \cline{2-12}

        & Fachka et al.~\cite{2012fachka}
        &  &  & 
        & NT-7
        & 2D & 2011
        & --- & ---
        & 
        & \\
        \midrule

        % Guillot et al.~\cite{2019guillot} 
        % & \cmark &
        % & NT-1
        % & 9Y & 2009--2018
        % & --- & ---
        % & \cite{2013quan}
        % & A3 \\

        % Enyanet et al.~\cite{2024enyanet} 
        % & &
        % &
        % & &
        % & &
        % & &
        % \\

        & \textbf{Aggregate}
        & 4/16 & 7/16 & ---
        & 8--24
        & 1w--3.5y & 2012--2021
        & --- & ---
        & 7/16
        & --- \\
        \bottomrule
    \end{tabular}
\end{table*}
\end{comment}
\begin{table*}[t!]
	\small
	\caption{
        Summary of network telescopes referenced in surveyed works. 
        Telescope size of historical datasets may vary as sizes listed reflect most recently reported values.
    }\label{tab:telescopes}
	\begin{tabular}{cccc}
		\toprule
		Telescopes & Country & Size & Data availability\\
		\midrule
		\textbf{NT-1.} UCSD-NT~\cite{ucsd-nt} & US & /9+/10 & Raw traces and flow data \\
		\textbf{NT-2.} Merit ORION~\cite{orion-nt} & US & $\sim$7 $\times$/16s & Raw trace, custom-defined event data\\
		\textbf{NT-3.} NICTER~\cite{nicter-nt} & JP &  /17, /18, 2$\times$/20 & TCP SYN only, anonymized flow data\\
		\textbf{NT-4.} Politecnico di Torino \cite{2020soro} & IT & 3 $\times$ /24 & Unknown\\
		\textbf{NT-5.} Darknet-BR \cite{2025camargo} & BR & /19 & Unknown\\
		\textbf{NT-6.} LHS Nancy~\cite{inria-nt} & FR & /20 & Raw trace\\
		\textbf{NT-7.} Farsight~\cite{farsight} & US & /13 & Unknown\\
        \textbf{NT-8.} TU Delft~\cite{delft-nt} & NL & /16 & Unknown\\
        \textbf{GT-1.} Deutsche Telekom AG~\cite{2020cohen} & DE & $\sim$/22 & Unknown\\
		\bottomrule
	\end{tabular}
\end{table*}

To demonstrate the utility of their detection methodologies, prior works conduct assessments using real-world darknet traffic datasets.
These assessments and their results provide evidence of a method's performance under realistic settings and offer empirical insights into its detection capabilities and general scalability.
In this section, we characterize prior works along the lines of credibility and reproducibility of their assessments by reviewing:
(i) characteristics of the darknet traffic datasets they use; and
(ii) their approach to validating detection outputs of their methodology.
Furthermore, we consider implementation details of detection methodologies since the reproducibility of assessments and results are closely tied to the availability of method source code and specifications of chosen computational environments.


% \vspace{0.25em}
\subsection{Characteristics of darknet traffic datasets.}

We consider three characteristics of each dataset used in prior works to assess detection methodologies: 
1) its timeframe;
2) the darknet it sources traffic;
and 3) its volume measured in units of packets and bytes.
Generally, the timeframes and darknet source bound a dataset's volume as more recent timeframes correlate 
with larger volumes as do larger darknets.
We use the first two characteristics to differentiate the datasets used in assessments while the third provides 
evidence of the scalability of a method assessed on a specific dataset.
% We consider three characteristics of each dataset used in prior work to assess detection methodologies.
% These include 1) the timeframe covered by the dataset, 2) the darknet whose traffic comprises the dataset, 3) and the dataset's traffic volume measured in units of packets and bytes.
% For the datasets used in prior works, we consider several characteristics relevant to interpreting method assessment results:
% the timeframe over which traffic was collected, 
% the specific darknet that received this traffic, 
% the size of the darknet's address space, and its traffic volume measured in units of packets and bytes.
% Empirical studies such as~\cite{2025camargo,2019soro} established that address space size has a non-trivial impact on the observability of darknet traffic (\textit{i.e.,} in terms of traffic composition and visibility of sources).

Across our surveyed works, we identified a total of 7 darknets (listed in Table~\ref{tab:telescopes}) from which datasets source their traffic. 
The amount of IPv4 address space occupied by each darknet varies; the largest and smallest sizes respectively span roughly a /9 (approx. 12.5M addresses in NT-1) 
and three /24 networks (768 addresses in NT-4). 
While these numbers represent the most up-to-date sizes reported by operators, darknet address space in practice may fluctuate based on ad-hoc BGP announcements 
(e.g., for leasing purposes~\cite{2025mannel} or in the case of more permanent changes such as address ownership transfers~\cite{2019ardc}). 
Empirical studies such as~\cite{2025camargo,2019soro} show that the size and ranges of IP addresses occupied by a darknet have a non-trivial impact on the observability, i.e., in terms of traffic composition and source visibility) of small-scale scanning events.
% Empirical studies such as~\cite{2025camargo,2019soro} established that address space size has a non-trivial impact on the observability of darknet traffic (\textit{i.e.,} in terms of traffic composition and visibility of sources).
While the effect that a darknet's physical geolocation on its observability has not been extensively studied, 
route topology situated between traffic sources and darknet address space likely play a non-negligible role.
In sum, the makeup of traffic varies darknet-to-darknet; assessment results closely tie to its 
specific selection of dataset and thus a method's performance may not generalize to other datasets.

We visualize the combined range of timeframes covered by darknet-traffic datasets from the start of 2012 until late 2023 in Figure~\ref{fig:cov_assess}. 
The proportion of time covered over this range by at least a single dataset (at a daily granularity) sits at \revise{68\%}. 
Most datasets are short snapshots: over half of all datasets span fewer than two months, roughly a third cover under three years of darknet traffic, and less than a fifth cover three or more years. 
Since a majority of works do not provide clear rationale for their timeframe selection, we infer that selection is largely opportunistic and based on availability of darknet traffic data at the time of assessment. 

However, some exceptions~\cite{2022kallitsis,2023kartsioukas} use reported dates of landmark 
Internet-wide scale events, \textit{e.g.}, the launch of the Mirai botnet, to guide their selection of timeframes.
We include 16 additional events (listed in Table~\ref{tab:events}) identified as either 
major remote execution vulnerabilities or scans originated by growing botnets 
whose traffic, in theory, should have been observed by darknets based on the 
reported scale of these events that range from hundreds of thousands to millions of affected hosts.
Although we find no explicit study of their detectability in darknet traffic in works other than~\cite{2022kallitsis,2023kartsioukas} 
we surveyed, 13 out of 17 events intersect with timeframes covered by at least two distinct darknet sources 
as shown in Figure~\ref{fig:cov_assess}. 
This finding has relevance for defining traffic labels which we further discuss in Section~\ref{sec:fw}.

% Some exceptions~\cite{2022kallitsis,2023kartsioukas} select specific timeframes centered around landmark Internet-wide events, e.g., the Mirai botnet, for validation purposes (later discussed in Section~\ref{sec:fw}).
% In addition to Mirai, Figure~\ref{fig:cov_assess} plots the start date of X additional events (summarized in Table~\ref{tab:events}), observable in darknet traffic, identified as either major vulnerabilities or scans by botnets.
% While X\% of these events are covered by dataset timeframes, none are mentioned explicitly in surveyed works besides the exceptions previously mentioned which further strengthens our inference. 
% We discuss another implication for future work in Section~\ref{sec:fw}.
% Recompute
% Nonetheless, darknet size directly correlates with traffic volumes and ostensibly more variability in its mixtures; thus the generalizability of method performance is sensitive to 
% larger darknets tend to receive greater and more varied traffic~\cite{@@} 

% \vspace{0.25em}
\mybox{Takeaways and Findings}{green!40}{green!10}{

\textbf{Takeaways:} 
Datasets used throughout method assessments possess different levels of observability in their traffic 
due to the variety of darknets and timeframes they sample.
Generalizability of a specific method's performance on one dataset to another remains unclear. 
}
\begin{table}[ht!]
    \caption{Historic landmark events either purported or confirmed to have been observed by darknets.}
    \label{tab:events}
    \centering
    \begin{tabular}{l c r l : l c r l}
    \hline
    \textbf{ID} & \textbf{Date} & \textbf{Event Name} & \textbf{Type} &
    \textbf{ID} & \textbf{Date} & \textbf{Event Name} & \textbf{Type} \\
    \hline
    A  & 2012-04 & Carna            & Botnet        &
    J  & 2017-10 & Reaper           & Botnet \\
    B  & 2014-03 & Heartbleed       & Remote Exp.   &
    K  & 2018-05 & VPNFilter        & Botnet \\
    C  & 2014-09 & Shellshock       & Remote Exp.   &
    L  & 2018-06 & Crackonosh       & Botnet \\
    D  & 2016-03 & Amnesia          & Botnet        &
    M  & 2019-05 & BlueKeep         & Remote Exp. \\
    E  & 2016-08 & Mirai            & Botnet        &
    N  & 2020-03 & SMBGhost         & Remote Exp. \\
    F  & 2017-04 & Eternal Blue     & Remote Exp.   &
    O  & 2020-06 & Ripple20         & Remote Exp. \\
    G  & 2017-04 & BrickerBot       & Botnet        &
    P  & 2021-06 & PrintNightmare   & Remote Exp. \\
    H  & 2017-05 & WannaCry         & Botnet        &
    Q  & 2021-12 & Log4JShell       & Remote Exp. \\
    I  & 2017-06 & NotPetya         & Botnet        &
       &         &                  &  \\
    \hline
    \end{tabular}
\end{table}
\begin{figure}[ht!]
    \centering
    \includegraphics[width=\textwidth]{figures/coverage_graph.pdf}
    \caption{Timeframes and sources of darknet-traffic datasets used in assessments (Section~\ref{sec:assessments}) of detection methodologies, overlapped with dates of Internet-wide events (listed in Table~\ref{tab:events}) observable by darknets.}
    \label{fig:cov_assess}
\end{figure}

% \vspace{0.25em}
% \noindent{\textbf{Validation of detection results.}}
\subsection{Strategies for validating detection results}

% Since limited definitive "ground truth" exists for attributing darknet traffic to their root causes, prior works rely on improvised strategies to validate their method outputs. 
% The defining practices of these strategies include whether they validate method outputs against labeled data, whether they incorporate external datasets 
% (as used in their label definitions or more generally to cross-validate results), and whether outputs from baseline methods are used as a reference point for comparison.
% We review these components, indicated in Table~\ref{tab:eval} for each work's validation strategy and consider the soundness and robustness of the overall approach.

Since limited definitive "ground truth" exists for attributing darknet traffic to their root causes, prior works rely on improvised strategies to validate method outputs.
As a defining component of these strategies, we review the types of information sources used to corroborate method outputs. These range from human-labeled data, external datasets (as used in label definitions or more generally to cross-validate), and detection outputs generated by baseline methods,.
We outline several groups that share common approaches based on their use of these components (listed in Table~\ref{tab:eval}) and discuss the scalability of their practices.

Among our surveyed works, 7~\cite{2019evrard,2015nishikaze,2019iglesias,2016ban,2020han,2019kanehara,2023kartsioukas} 
employ strategies designed to either confirm the results of exploratory analyses or verify 
that lack practices such as use of labeled data, cross-validation with external datasets, and comparisons against baseline methods.
designed to confirm the results of exploratory analyses.
A majority of these strategies lack the use of labeled data, cross-validation with external datasets, and comparisons against baseline methods.
Furthermore, their scope is limited in-practice by the volume of method outputs and the amount of manual investigation efforts 
(\textit{e.g.,} application of domain knowledge to interpret network traffic behavior, correlating outputs with third-party reports) required for validation.
Most works~\cite{2019evrard,2015nishikaze,2019iglesias,2016ban,2020han,2019kanehara,2023kartsioukas} investigate a limited number of outputs, 
linked to targeted ports of botnet scans~\cite{linuxmoose} and remotely exploitable vulnerabilities~\cite{redis,adbminer,memcached}.
More extensive validation is conducted by~\cite{2020han,2019iglesias}, respectively classifying 1,634 alerts and inspecting 20 traffic clusters. 

% The remaing X surveyed works adopt validation strategies that we consider more robust.
% Several works~\cite{2019lagraa,2022zakroum,2020soro,2014bouharb}
% feature the use of external datasets to improve the credibility of their results. 
% For example,  Zakroum et al.~\cite{2022zakroum} correlate public vulnerability reports with anomalous traffic activity on specific TCP ports detected by their method.
% Bou-harb et al.~\cite{2014bouharb} resolve IP addresses of their highest-volume senders against passively-collected DNS data to infer whether scans originated from malicious domains.
% While they lack the use of external datasets, another set of works~\cite{2017ban,2021han} instead center their validation strategy on comparisons against baseline methods.
% A final group of works~\cite{2022kallitsis,2023zakroum,2019bouharb,2023han,2025abduaziz} rely on validation strategies that we consider the most robust.

Validation strategies employed by the remainder of works~\cite{2019lagraa,2022zakroum,2020soro,2014bouharb,2017ban,2021han,2022kallitsis,2023zakroum,2019bouharb,2022han,2025abduaziz} 
adopt practices that are more sound compared to those previously discussed.
These practices enlarge the scope of validation efforts and strengthen the credibility of results 
through cross-validation with external datasets, (\textit{e.g.}, national vulnerability disclosures~\cite{nistnvd}, passively-collected domain names~\cite{farsight}), 
and comparisons against baseline methodologies.
Of these, we highlight several works~\cite{2022kallitsis,2023zakroum,2019bouharb,2022han,2025abduaziz} specifically for the rigor of their strategies that combine all three practices.
Label definitions for traffic are explicit and well-defined (\textit{e.g.,} malware fingerprints~\cite{2019ceron} and publicly known addresses of research projects and search engines).

\mybox{Takeaways and Findings}{green!40}{green!10}{
\textbf{Takeaways:}
Improvised strategies that prior works use to validate the results of method assessments vary by the scope and soundness of their practices.
Nonetheless, we found a small subset of works whose strategies serve as a standard for the degree of rigor that validation strategies should aim towards.
% We find little uniformity across the defining factors of improvised strategies used to validate method assessments. 
% However, we find a small subset of works that set a standard for designing a robust validation strategy in the absence of defintiive ground-truth.
% \textbf{Takeaways:} 
% Most darknet detection methods operate under severe ground-truth limitations, resulting in a spectrum of validation rigor. 
% Internal-only validation remains the norm, weak external validation is common but inconsistent, and strong validation—while the most convincing—is rare. 
% As a consequence, it is often difficult to determine whether a method detects genuine Internet events or artifacts of the dataset and evaluation procedure.
}

\label{sec:assess:rep}
\subsection{Replicability of assessments}
We consider the replicability of method assessments based on whether prior works provide public access to:
1) the datasets used in experiments; and 
2) implementation of methods (as source code or their software artifacts) along with specifications of the computing environment used to run experiments.
These two elements are essential for reproducing published results and applying existing methods to new traffic datasets.
We consider a methodology replicable only when both conditions are satisified.
% To determine the replicability of each detection methodology, we assess whether prior works provide
% 1) public access to their implementation, such as source code or software artifacts; 
% and 2) specifications of the computing environment used to run experiments. 
% These two elements are essential for reproducing published results and for applying existing techniques to new traffic datasets. 
% We consider a methodology replicable only when both conditions are satisfied.

Less than a third of surveyed works provide source code of their implementations and roughly half detail specifications of their experimental computing environments.
Of readily-available source code, we found Python and R as the programming languages of choice. Both are popular among numeric and scientific communities given their ease-of-use for implementing analytics workflows and abundant open-source algorithmic libraries.
Other works such as ~\cite{2019lagraa,2014bouharb} rely on Java or C, which have become less popular as preferences shift towards higher-languages for analytics use-cases.

Environments used to run experiments range from personal laptop workstations~\cite{2025abduaziz,2017lagraa,2019lagraa} to small-scale research compute clusters~\cite{2022han,2019bouharb,2022zakroum}.
Cluster sizes do not exceed five servers, individually equipped with at most 256GB of memory.
Most implementations execute strictly on CPUs while a subset of works that leverage representation-learning techniques employ GPUs for model training and evaluation.
Across our surveyed works, release dates of the CPUs and GPUs trail publication dates by as much as 10 years.

\mybox{Takeaways and Findings}{green!40}{green!10}{

\textbf{Takeaways:} 
    Fewer than a third of the proposed detection methodologies qualify as replicable under our two criteria.
    However, these that do are implemented using widely-adopted programming languages and leverage accessible hardware for experiment execution.
}

% \section{A Need for A Systemized Evaluation of Darknet Event Detection Methodologies}
\section{Challenges to a comprehensive assessment of methods}

Our survey of the current assessments of existing methods revealed inconsistencies  
across their datasets, implementations, and strategies used to validate their results.
In this section, we elaborate on these limitations and discuss how they hinder 
comprehensive assessment. 

\vspace{0.25em}
\noindent{\textbf{Limited replicability of assessments.}} 
Despite our findings reported in Section~\ref{sec:assess:rep}, we found low reusability for source code released by the small fraction of works. 
Our own atttempts to execute this code resulted in failures, remediated only with substantial modifications. 
While the primary goal of these studies is to demonstrate and disseminate research ideas rather than deliver production-grade software, 
the quality of the code introduces friction for anyone attempting to reproduce results. 
Cumulatively, this effect produces a replicability gap that to address, requires attention and thought to the quality of released software.

\vspace{0.25em}
\noindent{\textbf{Few direct comparative assessments.}}
% Fewer than half of our surveyed works directly compare detection methodologies via evaluations on the same dataset.
A second constraint arises from the scarcity of direct comparisons between detection methodologies. 
In the ideal case, competing approaches would be evaluated on a shared dataset, enabling clear, controlled comparisons of their relative capabilities. 
Yet fewer than half of the surveyed works perform such within-dataset comparisons, and only a handful evaluate multiple detection frameworks side-by-side using the same traffic. 
As a result, the literature offers limited evidence about how methods relate to each other empirically. 
Instead, most evaluations are siloed within individual studies, making it difficult to reason about comparative performance, robustness, or applicability across deployment contexts. 
In several instances, no more than two frameworks have ever been demonstrated on the same underlying dataset, leaving open questions about how methodological differences translate into practical differences in detection outcomes.

% Evaluations that directly compare detection methodologies on the same dataset yield the most definitive results of their relative capabilities.
% Fewer than half of surveyed works fit this criteria 
% Evaluation of detection methodologies by directly comparing their Direct comparisons of detection methodologies yield
% Evaluation of detection methodologies  yield definitive results.
% Slightly fewer than half of our surveyed works evaluate their proposed methodology against a baseline methodology using the same dataset.
% Of those, use of a dataset does not extend beyond an individual work, i.e., real-world traffic datasets are not widely shared for researchers to evaluate their methodologies.

% \begin{enumerate}
%     \item direct comparisons of detection methodologies by evaluating them on the same dataset is the ideal scenario
%     \item Slightly fewer than half of surveyed works fit this criteria
% 	\item we can't reasonably compare frameworks by their performance on different datasets.
% 	\item Few works compare different frameworks on the same dataset
% 	\item What's the highest number of frameworks that use the same dataset?
% \end{enumerate}

\vspace{0.25em}
\noindent{\textbf{Non-overlapping datasets used across assessments.}}
The challenge of comparison is further compounded by the heterogeneity of datasets used across studies. 
There is very little overlap in the traffic traces employed to evaluate frameworks: timeframes rarely coincide, traffic volumes often differ by multiple orders of magnitude, and data originate from distinct darknet deployments, network telescope sizes, and geographic vantage points. 
These non-overlapping datasets obscure whether observed performance reflects properties of the methodology or idiosyncrasies of the underlying traffic. 
Even when studies aim to detect similar classes of events, variation in temporal scope, probe density, and background traffic composition complicates any meaningful cross-paper interpretation. 
Collectively, this fragmentation of datasets reinforces the difficulty of establishing a unified baseline for evaluating darknet-based detection methods and highlights the need for shared, consistently curated datasets to support future methodological comparison.

\label{sec:fw}
\section{Future Work}

We conclude this report by describing several directions for future work to overcome the barriers that hinder a complete comparative assessment of detection methods. 

\vspace{0.25em}
\noindent{\textbf{Curated darknet traffic datasets.}}
One way to lower the barrier to conducting such assessments is to improve the availability of curated reference datasets.
Such curation involves intentional selection of darknet traffic and enrichment using well-defined labels to assess method performance.
These label definitions may include the malicious senders reported by IP blocklists~\cite{abuseipdb,firehol}, 
known scanning organizations~\cite{2023collins}, 
and packet fingerprints tied to scanning tools~\cite{2024griffioen} or malware~\cite{2017antonakakis}.

% Sound benchmarking practice requires assessing method implementations on shared datasets. 
% By providing curated darknet-traffic datasets as inputs to our benchmarking framework, 
% we ensure that differences across outputs stem from the methods themselves rather than from inconsistencies in the data.
% The curation process involves intentional selection of traffic from specific darknets, sampled over historical time periods, ideally rich with known, landmark events. 
% Futher, labels accompany the traffic. \maxnote{elaborate}

\vspace{0.25em}
\noindent{\textbf{Consistent method implementations.}}
Consistency across the software used to implement methods and the execution environments that run assessments 
ensures the validity of performance metrics and improves interpretability of results.
While in-practice this is difficult to achieve when multiple parties reproduce assessments,
more thorough documentation (\textit{e.g.,} software libraries, CPUs, GPUs, memory, IO, and storage hardware components) 
can substitute for up to a degree of inconsistency.
% not only ensure the validity of computational performance results between different methods, 
% but also improves overall reproducibility of assessments by reducing the friction of executing source code.
% Practically, this may involve documentation of execution environments (\textit{e.g.}, CPUs, GPUs, memory, IO, and storage components)
% and alignment.

\vspace{0.25em}
\noindent{\textbf{Standardized validation strategies.}}
Enabled by labeled reference datasets and consistent method implementations, 
standard metrics to assess detection and computational performance are components of a standardized 
approach crucial to interpretation of assessment results.

% Enabled by assessments using shared reference datasets,
% conventional metrics (\textit{e.g.,} true/false positives/negatives and their derivative measures) 
% facilitate comparisons of method performance on labeled classes of traffic. 
% On the other hand, differential metrics (\textit{e.g.,} jaccard index) support comparisons 
% assessments using unlabeled traffic data. 

% \section{Acknowledgments}
% asadfadsf
% \section{Appendices}

\begin{table}[]
    \small
    \caption{Algorithms used by each surveyed framework.}
    \label{tab:frameworks-algorithms}
    % Use p{<width>} to allow wrapping of long algorithm lists.
    \begin{tabular}{lp{0.62\linewidth}}
        \toprule
        \textbf{Work} & \textbf{Algorithm(s)} \\
        \midrule
        Evrard et al.~\cite{2019evrard}                        & Dijkstra's~\cite{1959dijkstra}; K-NN~\cite{1967cover,1989fix} \\
        Lagraa et al.~\cite{2017lagraa,2019lagraa}             & Louvain~\cite{2006newman,2008blondel} \\
        Kallitsis et al.~\cite{2022kallitsis}                  & Autoencoder Dimensionality Reduction~\cite{2006hinton}; K-Means~\cite{1967macqueen} \\
        Iglesias et al.~\cite{2019iglesias}                    & K-Medoids~\cite{2009park}; Fuzzy-Gustafson~\cite{1999krishnapuram}; MAD-Thresholding~\cite{2004liu} \\
        Nishikaze et al.~\cite{2015nishikaze}                  & Hierarchical Clustering~\cite{2012murtagh} \\
        Soro et al.~\cite{2020soro}                            & Louvain Algorithm~\cite{2006newman,2008blondel} \\
        Gioacchini et al.~\cite{2021gioacchini,2023gioacchini} & Word2Vec~\cite{2013mikolov}; K-Means~\cite{1967macqueen}; K-NN~\cite{1967cover,1989fix}; Louvain~\cite{2006newman,2008blondel} \\
        Abduaziz et al.~\cite{2025abduaziz}                    & Word2Vec~\cite{2025abduaziz}; HDBScan~\cite{2013campello,2018gertrudes} \\
        Han et al.~\cite{2021han,2022han}                      & NMF~\cite{2000lee} \\
        Han et al.~\cite{2020han,2022han}                      & GLASSO~\cite{2008friedman} \\
        Kanehara et al.~\cite{2019kanehara,2022han}            & LRA-NTD~\cite{2015zhou}; FTSD~\cite{2010caiafa}; Otsu-Thresholding~\cite{1979otsu} \\
        Kartsioukas et al.~\cite{2023kartsioukas}              & Incremental PCA~\cite{2012arora} \\
        Ban et al.~\cite{2016ban}                              & Frequent Pattern Mining~\cite{2000han,2007han}; Hierarchical Clustering~\cite{2012murtagh} \\
        Torabi et al.~\cite{2020torabi,2018torabi}             & Association Rule Mining~\cite{1993agrawal}; DBSCAN~\cite{1996ester} \\
        % Tanaka et al.~\cite{2023tanaka,2021tanaka}             & \textit{TODO} \\
        Niranjana et al.~\cite{2019niranjana}                  & PCA~\cite{1901pearson,1993hotelling} \\
        Cabana et al.~\cite{2019cabana}                        & Conduction Detection Algorithm~\cite{2015lu}; Fastcluster~\cite{2013mullner} \\
        % Shaikh et al.~\cite{2018shaikh}                        & AdaBoost~\cite{@@}; Gradient Boosting~\cite{@@}; Random Forest~\cite{@@} \\
        Zakroum et al.~\cite{2022zakroum,2018zakroum}          & Spectral Clustering~\cite{2001ng}; LSTM~\cite{1997hochreiter} \\
        \bottomrule
    \end{tabular}
\end{table}


% \begin{acks}
% To Robert, for the bagels and explaining CMYK and color spaces.
% \end{acks}

\bibliographystyle{plain}
\bibliography{bib/refs.bib}

% \appendix
% \section{Research Methods}

% \subsection{Part One}

% Lorem ipsum dolor sit amet, consectetur adipiscing elit. Morbi
% malesuada, quam in pulvinar varius, metus nunc fermentum urna, id
% sollicitudin purus odio sit amet enim. Aliquam ullamcorper eu ipsum
% vel mollis. Curabitur quis dictum nisl. Phasellus vel semper risus, et
% lacinia dolor. Integer ultricies commodo sem nec semper.

\end{document}
\endinput